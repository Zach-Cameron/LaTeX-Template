\section{Section}

\lipsum[1-2]

%%%%%%%%%%%%%%%%%%%%%%%
\subsection{Subsection}

\lipsum[][1-9]

\begin{defn}[\ind{Defn Ipsum}]
    \lipsum[][1] \(\sum_{i=1}^{n} n = \frac{n(n+1)}{2}\) \lipsum[][2-3]
    \[e^{i\pi}+1 = 0\]
    \lipsum[][4] \(\int_{a}^{b} x^2 \ dx = \frac{x^3}{3} + C\) \lipsum[][5-6]
\end{defn}

\begin{nota}
    \lipsum[][1-3]
    \[\sum_{i=1}^{n} n = \frac{n(n+1)}{2}\]
    \lipsum[][4-5]
    \[\int x \ dx = \frac{x^2}{2} + C\]
    \lipsum[][6-7]
\end{nota}

\begin{defn}
    %Given any natural number \(n\):
    \[\sum_{i=1}^{n} n = \frac{n(n+1)}{2}\]
    \lipsum[][3-5]
\end{defn}

\begin{rmrk}
    No environment should ever start with inline- or (especially not) display-mode math. Not only is that \href{https://kconrad.math.uconn.edu/blurbs/proofs/writingtips.pdf}{bad writing practice} but, in the case of starting with display-mode math such as above, blank vertical space will be left before the display-mode math where text is expected to be.

    If you don't know what to write, just state some context about the equation or expression with which you intended to start, e.g. `Given any natural number \(n\):'.
\end{rmrk}

\begin{exmp}[This is transcribed from Example 1.3 \href{https://kconrad.math.uconn.edu/blurbs/proofs/welldefined.pdf}{here}]
    In calculus, \(\int_{a}^{b} f(x) \ dx\) can be computed as
    \[\int_{a}^{b} f(x) \ dx = F(b) - F(a),\]
    where \(F(x)\) is an arbitrary anti-derivative of \(f(x)\) on \([a,b]\), \textit{i.e.}, \(F'(x) = f(x)\) for all \(x\) in \([a,b]\). This formula for \(\int_{a}^{b} f(x) \ dx\) involves a choice of anti-derivative for \(f(x)\), but the formula does \textit{not} depend on the choice: every anti-derivative \(G(x)\) of \(f(x)\) on \([a,b]\) differs from \(F(x)\) by a constant, say \(G(x) = F(x) + C\) for all \(x\) in \([a,b]\), and changing the anti-derivative \(G(x)\) does not change the difference of its values at the endpoints:
    \[G(b) - G(a) = (F(b) + C) - (F(a) + C) = F(b) - F(a).\]
    So the difference of the values of an anti-derivative of \(f(x)\) at \(x=a\) and \(x=b\) is independent of the choice of anti-derivative of \(f(x)\) on the interval \([a,b]\).\footnote[2]{This is why in physics, potential energy has no intrinsic meaning (the zero level of potential energy can be anywhere), but \textit{differences} in potential energy are physically meaningful.}

    In contrast, the ``rule'' \(F(b)+F(a)\) depends on the choice of anti-derivative of \(f(x)\), since
    \[G(b)+G(a) = (F(b)+ C) + (F(a) + C) = F(b) + F(a) + 2C,\]
    which is a new value if \(C \neq 0\). Taking differences in an anti-derivative cancels the effect of the undetermined additive constant, so the expression \(F(b)-F(a)\) is a well-defined value based on the original input function \(f(x)\) and the interval \([a,b]\).
\end{exmp}

\begin{note}
    \lipsum[][1-9]
\end{note}

\begin{rmrk}
    \lipsum[][1-9]
\end{rmrk}

\begin{lemm}
    \lipsum[][1-4]
\end{lemm}

\begin{proof}
    \lipsum[1]
\end{proof}

\begin{thrm}[\ind{Thrm Ipsum}]
    \lipsum[][1-4]
\end{thrm}

\begin{proof}
    \lipsum[1]
\end{proof}

\begin{coro}
    \lipsum[][1-4]
\end{coro}

\begin{exrc}
    \lipsum[1]
\end{exrc}

\begin{soln}
    \lipsum[1]
\end{soln}

%%%%%%%%%%%%%%%%%%%%%%%%%%%%%
\subsubsection{Subsubsection}

\begin{defn}[\ind{Defn Ipsum}]
    \lipsum[][1] \(\sum_{i=1}^{n} n = \frac{n(n+1)}{2}\) \lipsum[][2-3]
    \[e^{i\pi}+1 = 0\]
    \lipsum[][4] \(\int_{a}^{b} x^2 \ dx = \frac{x^3}{3} + C\) \lipsum[][5-6]
\end{defn}

\begin{nota}
    \lipsum[][1-3]
    \[\sum_{i=1}^{n} n = \frac{n(n+1)}{2}\]
    \lipsum[][4-5]
    \[\int x \ dx = \frac{x^2}{2} + C\]
    \lipsum[][6-7]
\end{nota}

\begin{defn}
    %Given any natural number \(n\):
    \[\sum_{i=1}^{n} n = \frac{n(n+1)}{2}\]
    \lipsum[][3-5]
\end{defn}

\begin{rmrk}
    No environment should ever start with inline- or (especially not) display-mode math. Not only is that \href{https://kconrad.math.uconn.edu/blurbs/proofs/writingtips.pdf}{bad writing practice} but, in the case of starting with display-mode math such as above, blank vertical space will be left before the display-mode math where text is expected to be.

    If you don't know what to write, just state some context about the equation or expression with which you intended to start, e.g. `Given any natural number \(n\):'.
\end{rmrk}

\begin{exmp}[This is transcribed from Example 1.3 \href{https://kconrad.math.uconn.edu/blurbs/proofs/welldefined.pdf}{here}]
    In calculus, \(\int_{a}^{b} f(x) \ dx\) can be computed as
    \[\int_{a}^{b} f(x) \ dx = F(b) - F(a),\]
    where \(F(x)\) is an arbitrary anti-derivative of \(f(x)\) on \([a,b]\), \textit{i.e.}, \(F'(x) = f(x)\) for all \(x\) in \([a,b]\). This formula for \(\int_{a}^{b} f(x) \ dx\) involves a choice of anti-derivative for \(f(x)\), but the formula does \textit{not} depend on the choice: every anti-derivative \(G(x)\) of \(f(x)\) on \([a,b]\) differs from \(F(x)\) by a constant, say \(G(x) = F(x) + C\) for all \(x\) in \([a,b]\), and changing the anti-derivative \(G(x)\) does not change the difference of its values at the endpoints:
    \[G(b) - G(a) = (F(b) + C) - (F(a) + C) = F(b) - F(a).\]
    So the difference of the values of an anti-derivative of \(f(x)\) at \(x=a\) and \(x=b\) is independent of the choice of anti-derivative of \(f(x)\) on the interval \([a,b]\).\footnote[2]{This is why in physics, potential energy has no intrinsic meaning (the zero level of potential energy can be anywhere), but \textit{differences} in potential energy are physically meaningful.}

    In contrast, the ``rule'' \(F(b)+F(a)\) depends on the choice of anti-derivative of \(f(x)\), since
    \[G(b)+G(a) = (F(b)+ C) + (F(a) + C) = F(b) + F(a) + 2C,\]
    which is a new value if \(C \neq 0\). Taking differences in an anti-derivative cancels the effect of the undetermined additive constant, so the expression \(F(b)-F(a)\) is a well-defined value based on the original input function \(f(x)\) and the interval \([a,b]\).
\end{exmp}

\begin{note}
    \lipsum[][1-9]
\end{note}

\begin{rmrk}
    \lipsum[][1-9]
\end{rmrk}

\begin{lemm}
    \lipsum[][1-4]
\end{lemm}

\begin{proof}
    \lipsum[1]
\end{proof}

\begin{thrm}[\ind{Thrm Ipsum}]
    \lipsum[][1-4]
\end{thrm}

\begin{proof}
    \lipsum[1]
\end{proof}

\begin{coro}
    \lipsum[][1-4]
\end{coro}

%%%%%%%%%%%%%%%%%%%%%%%%%%%%%
\subsubsection{Subsubsection}

\begin{defn}[\ind{Defn Ipsum}]
    \lipsum[][1] \(\sum_{i=1}^{n} n = \frac{n(n+1)}{2}\) \lipsum[][2-3]
    \[e^{i\pi}+1 = 0\]
    \lipsum[][4] \(\int_{a}^{b} x^2 \ dx = \frac{x^3}{3} + C\) \lipsum[][5-6]
\end{defn}

\begin{nota}
    \lipsum[][1-3]
    \[\sum_{i=1}^{n} n = \frac{n(n+1)}{2}\]
    \lipsum[][4-5]
    \[\int x \ dx = \frac{x^2}{2} + C\]
    \lipsum[][6-7]
\end{nota}

\begin{defn}
    %Given any natural number \(n\):
    \[\sum_{i=1}^{n} n = \frac{n(n+1)}{2}\]
    \lipsum[][3-5]
\end{defn}

\begin{rmrk}
    No environment should ever start with inline- or (especially not) display-mode math. Not only is that \href{https://kconrad.math.uconn.edu/blurbs/proofs/writingtips.pdf}{bad writing practice} but, in the case of starting with display-mode math such as above, blank vertical space will be left before the display-mode math where text is expected to be.

    If you don't know what to write, just state some context about the equation or expression with which you intended to start, e.g. `Given any natural number \(n\):'.
\end{rmrk}

\begin{exmp}[This is transcribed from Example 1.3 \href{https://kconrad.math.uconn.edu/blurbs/proofs/welldefined.pdf}{here}]
    In calculus, \(\int_{a}^{b} f(x) \ dx\) can be computed as
    \[\int_{a}^{b} f(x) \ dx = F(b) - F(a),\]
    where \(F(x)\) is an arbitrary anti-derivative of \(f(x)\) on \([a,b]\), \textit{i.e.}, \(F'(x) = f(x)\) for all \(x\) in \([a,b]\). This formula for \(\int_{a}^{b} f(x) \ dx\) involves a choice of anti-derivative for \(f(x)\), but the formula does \textit{not} depend on the choice: every anti-derivative \(G(x)\) of \(f(x)\) on \([a,b]\) differs from \(F(x)\) by a constant, say \(G(x) = F(x) + C\) for all \(x\) in \([a,b]\), and changing the anti-derivative \(G(x)\) does not change the difference of its values at the endpoints:
    \[G(b) - G(a) = (F(b) + C) - (F(a) + C) = F(b) - F(a).\]
    So the difference of the values of an anti-derivative of \(f(x)\) at \(x=a\) and \(x=b\) is independent of the choice of anti-derivative of \(f(x)\) on the interval \([a,b]\).\footnote[2]{This is why in physics, potential energy has no intrinsic meaning (the zero level of potential energy can be anywhere), but \textit{differences} in potential energy are physically meaningful.}

    In contrast, the ``rule'' \(F(b)+F(a)\) depends on the choice of anti-derivative of \(f(x)\), since
    \[G(b)+G(a) = (F(b)+ C) + (F(a) + C) = F(b) + F(a) + 2C,\]
    which is a new value if \(C \neq 0\). Taking differences in an anti-derivative cancels the effect of the undetermined additive constant, so the expression \(F(b)-F(a)\) is a well-defined value based on the original input function \(f(x)\) and the interval \([a,b]\).
\end{exmp}

\begin{note}
    \lipsum[][1-9]
\end{note}

\begin{rmrk}
    \lipsum[][1-9]
\end{rmrk}

\begin{lemm}
    \lipsum[][1-4]
\end{lemm}

\begin{proof}
    \lipsum[1]
\end{proof}

\begin{thrm}[\ind{Thrm Ipsum}]
    \lipsum[][1-4]
\end{thrm}

\begin{proof}
    \lipsum[1]
\end{proof}

\begin{coro}
    \lipsum[][1-4]
\end{coro}

%%%%%%%%%%%%%%%%%%%%%%%%%%%%%
\subsubsection{Subsubsection}

\begin{defn}[\ind{Defn Ipsum}]
    \lipsum[][1] \(\sum_{i=1}^{n} n = \frac{n(n+1)}{2}\) \lipsum[][2-3]
    \[e^{i\pi}+1 = 0\]
    \lipsum[][4] \(\int_{a}^{b} x^2 \ dx = \frac{x^3}{3} + C\) \lipsum[][5-6]
\end{defn}

\begin{nota}
    \lipsum[][1-3]
    \[\sum_{i=1}^{n} n = \frac{n(n+1)}{2}\]
    \lipsum[][4-5]
    \[\int x \ dx = \frac{x^2}{2} + C\]
    \lipsum[][6-7]
\end{nota}

\begin{defn}
    %Given any natural number \(n\):
    \[\sum_{i=1}^{n} n = \frac{n(n+1)}{2}\]
    \lipsum[][3-5]
\end{defn}

\begin{rmrk}
    No environment should ever start with inline- or (especially not) display-mode math. Not only is that \href{https://kconrad.math.uconn.edu/blurbs/proofs/writingtips.pdf}{bad writing practice} but, in the case of starting with display-mode math such as above, blank vertical space will be left before the display-mode math where text is expected to be.

    If you don't know what to write, just state some context about the equation or expression with which you intended to start, e.g. `Given any natural number \(n\):'.
\end{rmrk}

\begin{exmp}[This is transcribed from Example 1.3 \href{https://kconrad.math.uconn.edu/blurbs/proofs/welldefined.pdf}{here}]
    In calculus, \(\int_{a}^{b} f(x) \ dx\) can be computed as
    \[\int_{a}^{b} f(x) \ dx = F(b) - F(a),\]
    where \(F(x)\) is an arbitrary anti-derivative of \(f(x)\) on \([a,b]\), \textit{i.e.}, \(F'(x) = f(x)\) for all \(x\) in \([a,b]\). This formula for \(\int_{a}^{b} f(x) \ dx\) involves a choice of anti-derivative for \(f(x)\), but the formula does \textit{not} depend on the choice: every anti-derivative \(G(x)\) of \(f(x)\) on \([a,b]\) differs from \(F(x)\) by a constant, say \(G(x) = F(x) + C\) for all \(x\) in \([a,b]\), and changing the anti-derivative \(G(x)\) does not change the difference of its values at the endpoints:
    \[G(b) - G(a) = (F(b) + C) - (F(a) + C) = F(b) - F(a).\]
    So the difference of the values of an anti-derivative of \(f(x)\) at \(x=a\) and \(x=b\) is independent of the choice of anti-derivative of \(f(x)\) on the interval \([a,b]\).\footnote[2]{This is why in physics, potential energy has no intrinsic meaning (the zero level of potential energy can be anywhere), but \textit{differences} in potential energy are physically meaningful.}

    In contrast, the ``rule'' \(F(b)+F(a)\) depends on the choice of anti-derivative of \(f(x)\), since
    \[G(b)+G(a) = (F(b)+ C) + (F(a) + C) = F(b) + F(a) + 2C,\]
    which is a new value if \(C \neq 0\). Taking differences in an anti-derivative cancels the effect of the undetermined additive constant, so the expression \(F(b)-F(a)\) is a well-defined value based on the original input function \(f(x)\) and the interval \([a,b]\).
\end{exmp}

\begin{note}
    \lipsum[][1-9]
\end{note}

\begin{rmrk}
    \lipsum[][1-9]
\end{rmrk}

\begin{lemm}
    \lipsum[][1-4]
\end{lemm}

\begin{proof}
    \lipsum[1]
\end{proof}

\begin{thrm}[\ind{Thrm Ipsum}]
    \lipsum[][1-4]
\end{thrm}

\begin{proof}
    \lipsum[1]
\end{proof}

\begin{coro}
    \lipsum[][1-4]
\end{coro}

%%%%%%%%%%%%%%%%%%%%%%%
\subsection{Subsection}

\lipsum[][1-9]

\begin{defn}[\ind{Defn Ipsum}]
    \lipsum[][1] \(\sum_{i=1}^{n} n = \frac{n(n+1)}{2}\) \lipsum[][2-3]
    \[e^{i\pi}+1 = 0\]
    \lipsum[][4] \(\int_{a}^{b} x^2 \ dx = \frac{x^3}{3} + C\) \lipsum[][5-6]
\end{defn}

\begin{nota}
    \lipsum[][1-3]
    \[\sum_{i=1}^{n} n = \frac{n(n+1)}{2}\]
    \lipsum[][4-5]
    \[\int x \ dx = \frac{x^2}{2} + C\]
    \lipsum[][6-7]
\end{nota}

\begin{defn}
    %Given any natural number \(n\):
    \[\sum_{i=1}^{n} n = \frac{n(n+1)}{2}\]
    \lipsum[][3-5]
\end{defn}

\begin{rmrk}
    No environment should ever start with inline- or (especially not) display-mode math. Not only is that \href{https://kconrad.math.uconn.edu/blurbs/proofs/writingtips.pdf}{bad writing practice} but, in the case of starting with display-mode math such as above, blank vertical space will be left before the display-mode math where text is expected to be.

    If you don't know what to write, just state some context about the equation or expression with which you intended to start, e.g. `Given any natural number \(n\):'.
\end{rmrk}

\begin{exmp}[This is transcribed from Example 1.3 \href{https://kconrad.math.uconn.edu/blurbs/proofs/welldefined.pdf}{here}]
    In calculus, \(\int_{a}^{b} f(x) \ dx\) can be computed as
    \[\int_{a}^{b} f(x) \ dx = F(b) - F(a),\]
    where \(F(x)\) is an arbitrary anti-derivative of \(f(x)\) on \([a,b]\), \textit{i.e.}, \(F'(x) = f(x)\) for all \(x\) in \([a,b]\). This formula for \(\int_{a}^{b} f(x) \ dx\) involves a choice of anti-derivative for \(f(x)\), but the formula does \textit{not} depend on the choice: every anti-derivative \(G(x)\) of \(f(x)\) on \([a,b]\) differs from \(F(x)\) by a constant, say \(G(x) = F(x) + C\) for all \(x\) in \([a,b]\), and changing the anti-derivative \(G(x)\) does not change the difference of its values at the endpoints:
    \[G(b) - G(a) = (F(b) + C) - (F(a) + C) = F(b) - F(a).\]
    So the difference of the values of an anti-derivative of \(f(x)\) at \(x=a\) and \(x=b\) is independent of the choice of anti-derivative of \(f(x)\) on the interval \([a,b]\).\footnote[2]{This is why in physics, potential energy has no intrinsic meaning (the zero level of potential energy can be anywhere), but \textit{differences} in potential energy are physically meaningful.}

    In contrast, the ``rule'' \(F(b)+F(a)\) depends on the choice of anti-derivative of \(f(x)\), since
    \[G(b)+G(a) = (F(b)+ C) + (F(a) + C) = F(b) + F(a) + 2C,\]
    which is a new value if \(C \neq 0\). Taking differences in an anti-derivative cancels the effect of the undetermined additive constant, so the expression \(F(b)-F(a)\) is a well-defined value based on the original input function \(f(x)\) and the interval \([a,b]\).
\end{exmp}

\begin{note}
    \lipsum[][1-9]
\end{note}

\begin{rmrk}
    \lipsum[][1-9]
\end{rmrk}

\begin{lemm}
    \lipsum[][1-4]
\end{lemm}

\begin{proof}
    \lipsum[1]
\end{proof}

\begin{thrm}[\ind{Thrm Ipsum}]
    \lipsum[][1-4]
\end{thrm}

\begin{proof}
    \lipsum[1]
\end{proof}

\begin{coro}
    \lipsum[][1-4]
\end{coro}

\begin{exrc}
    \lipsum[1]
\end{exrc}

\begin{soln}
    \lipsum[1]
\end{soln}

%%%%%%%%%%%%%%%%%%%%%%%%%%%%%
\subsubsection{Subsubsection}

\begin{defn}[\ind{Defn Ipsum}]
    \lipsum[][1] \(\sum_{i=1}^{n} n = \frac{n(n+1)}{2}\) \lipsum[][2-3]
    \[e^{i\pi}+1 = 0\]
    \lipsum[][4] \(\int_{a}^{b} x^2 \ dx = \frac{x^3}{3} + C\) \lipsum[][5-6]
\end{defn}

\begin{nota}
    \lipsum[][1-3]
    \[\sum_{i=1}^{n} n = \frac{n(n+1)}{2}\]
    \lipsum[][4-5]
    \[\int x \ dx = \frac{x^2}{2} + C\]
    \lipsum[][6-7]
\end{nota}

\begin{defn}
    %Given any natural number \(n\):
    \[\sum_{i=1}^{n} n = \frac{n(n+1)}{2}\]
    \lipsum[][3-5]
\end{defn}

\begin{rmrk}
    No environment should ever start with inline- or (especially not) display-mode math. Not only is that \href{https://kconrad.math.uconn.edu/blurbs/proofs/writingtips.pdf}{bad writing practice} but, in the case of starting with display-mode math such as above, blank vertical space will be left before the display-mode math where text is expected to be.

    If you don't know what to write, just state some context about the equation or expression with which you intended to start, e.g. `Given any natural number \(n\):'.
\end{rmrk}

\begin{exmp}[This is transcribed from Example 1.3 \href{https://kconrad.math.uconn.edu/blurbs/proofs/welldefined.pdf}{here}]
    In calculus, \(\int_{a}^{b} f(x) \ dx\) can be computed as
    \[\int_{a}^{b} f(x) \ dx = F(b) - F(a),\]
    where \(F(x)\) is an arbitrary anti-derivative of \(f(x)\) on \([a,b]\), \textit{i.e.}, \(F'(x) = f(x)\) for all \(x\) in \([a,b]\). This formula for \(\int_{a}^{b} f(x) \ dx\) involves a choice of anti-derivative for \(f(x)\), but the formula does \textit{not} depend on the choice: every anti-derivative \(G(x)\) of \(f(x)\) on \([a,b]\) differs from \(F(x)\) by a constant, say \(G(x) = F(x) + C\) for all \(x\) in \([a,b]\), and changing the anti-derivative \(G(x)\) does not change the difference of its values at the endpoints:
    \[G(b) - G(a) = (F(b) + C) - (F(a) + C) = F(b) - F(a).\]
    So the difference of the values of an anti-derivative of \(f(x)\) at \(x=a\) and \(x=b\) is independent of the choice of anti-derivative of \(f(x)\) on the interval \([a,b]\).\footnote[2]{This is why in physics, potential energy has no intrinsic meaning (the zero level of potential energy can be anywhere), but \textit{differences} in potential energy are physically meaningful.}

    In contrast, the ``rule'' \(F(b)+F(a)\) depends on the choice of anti-derivative of \(f(x)\), since
    \[G(b)+G(a) = (F(b)+ C) + (F(a) + C) = F(b) + F(a) + 2C,\]
    which is a new value if \(C \neq 0\). Taking differences in an anti-derivative cancels the effect of the undetermined additive constant, so the expression \(F(b)-F(a)\) is a well-defined value based on the original input function \(f(x)\) and the interval \([a,b]\).
\end{exmp}

\begin{note}
    \lipsum[][1-9]
\end{note}

\begin{rmrk}
    \lipsum[][1-9]
\end{rmrk}

\begin{lemm}
    \lipsum[][1-4]
\end{lemm}

\begin{proof}
    \lipsum[1]
\end{proof}

\begin{thrm}[\ind{Thrm Ipsum}]
    \lipsum[][1-4]
\end{thrm}

\begin{proof}
    \lipsum[1]
\end{proof}

\begin{coro}
    \lipsum[][1-4]
\end{coro}

%%%%%%%%%%%%%%%%%%%%%%%%%%%%%
\subsubsection{Subsubsection}

\begin{defn}[\ind{Defn Ipsum}]
    \lipsum[][1] \(\sum_{i=1}^{n} n = \frac{n(n+1)}{2}\) \lipsum[][2-3]
    \[e^{i\pi}+1 = 0\]
    \lipsum[][4] \(\int_{a}^{b} x^2 \ dx = \frac{x^3}{3} + C\) \lipsum[][5-6]
\end{defn}

\begin{nota}
    \lipsum[][1-3]
    \[\sum_{i=1}^{n} n = \frac{n(n+1)}{2}\]
    \lipsum[][4-5]
    \[\int x \ dx = \frac{x^2}{2} + C\]
    \lipsum[][6-7]
\end{nota}

\begin{defn}
    %Given any natural number \(n\):
    \[\sum_{i=1}^{n} n = \frac{n(n+1)}{2}\]
    \lipsum[][3-5]
\end{defn}

\begin{rmrk}
    No environment should ever start with inline- or (especially not) display-mode math. Not only is that \href{https://kconrad.math.uconn.edu/blurbs/proofs/writingtips.pdf}{bad writing practice} but, in the case of starting with display-mode math such as above, blank vertical space will be left before the display-mode math where text is expected to be.

    If you don't know what to write, just state some context about the equation or expression with which you intended to start, e.g. `Given any natural number \(n\):'.
\end{rmrk}

\begin{exmp}[This is transcribed from Example 1.3 \href{https://kconrad.math.uconn.edu/blurbs/proofs/welldefined.pdf}{here}]
    In calculus, \(\int_{a}^{b} f(x) \ dx\) can be computed as
    \[\int_{a}^{b} f(x) \ dx = F(b) - F(a),\]
    where \(F(x)\) is an arbitrary anti-derivative of \(f(x)\) on \([a,b]\), \textit{i.e.}, \(F'(x) = f(x)\) for all \(x\) in \([a,b]\). This formula for \(\int_{a}^{b} f(x) \ dx\) involves a choice of anti-derivative for \(f(x)\), but the formula does \textit{not} depend on the choice: every anti-derivative \(G(x)\) of \(f(x)\) on \([a,b]\) differs from \(F(x)\) by a constant, say \(G(x) = F(x) + C\) for all \(x\) in \([a,b]\), and changing the anti-derivative \(G(x)\) does not change the difference of its values at the endpoints:
    \[G(b) - G(a) = (F(b) + C) - (F(a) + C) = F(b) - F(a).\]
    So the difference of the values of an anti-derivative of \(f(x)\) at \(x=a\) and \(x=b\) is independent of the choice of anti-derivative of \(f(x)\) on the interval \([a,b]\).\footnote[2]{This is why in physics, potential energy has no intrinsic meaning (the zero level of potential energy can be anywhere), but \textit{differences} in potential energy are physically meaningful.}

    In contrast, the ``rule'' \(F(b)+F(a)\) depends on the choice of anti-derivative of \(f(x)\), since
    \[G(b)+G(a) = (F(b)+ C) + (F(a) + C) = F(b) + F(a) + 2C,\]
    which is a new value if \(C \neq 0\). Taking differences in an anti-derivative cancels the effect of the undetermined additive constant, so the expression \(F(b)-F(a)\) is a well-defined value based on the original input function \(f(x)\) and the interval \([a,b]\).
\end{exmp}

\begin{note}
    \lipsum[][1-9]
\end{note}

\begin{rmrk}
    \lipsum[][1-9]
\end{rmrk}

\begin{lemm}
    \lipsum[][1-4]
\end{lemm}

\begin{proof}
    \lipsum[1]
\end{proof}

\begin{thrm}[\ind{Thrm Ipsum}]
    \lipsum[][1-4]
\end{thrm}

\begin{proof}
    \lipsum[1]
\end{proof}

\begin{coro}
    \lipsum[][1-4]
\end{coro}

%%%%%%%%%%%%%%%%%%%%%%%%%%%%%
\subsubsection{Subsubsection}

\begin{defn}[\ind{Defn Ipsum}]
    \lipsum[][1] \(\sum_{i=1}^{n} n = \frac{n(n+1)}{2}\) \lipsum[][2-3]
    \[e^{i\pi}+1 = 0\]
    \lipsum[][4] \(\int_{a}^{b} x^2 \ dx = \frac{x^3}{3} + C\) \lipsum[][5-6]
\end{defn}

\begin{nota}
    \lipsum[][1-3]
    \[\sum_{i=1}^{n} n = \frac{n(n+1)}{2}\]
    \lipsum[][4-5]
    \[\int x \ dx = \frac{x^2}{2} + C\]
    \lipsum[][6-7]
\end{nota}

\begin{defn}
    %Given any natural number \(n\):
    \[\sum_{i=1}^{n} n = \frac{n(n+1)}{2}\]
    \lipsum[][3-5]
\end{defn}

\begin{rmrk}
    No environment should ever start with inline- or (especially not) display-mode math. Not only is that \href{https://kconrad.math.uconn.edu/blurbs/proofs/writingtips.pdf}{bad writing practice} but, in the case of starting with display-mode math such as above, blank vertical space will be left before the display-mode math where text is expected to be.

    If you don't know what to write, just state some context about the equation or expression with which you intended to start, e.g. `Given any natural number \(n\):'.
\end{rmrk}

\begin{exmp}[This is transcribed from Example 1.3 \href{https://kconrad.math.uconn.edu/blurbs/proofs/welldefined.pdf}{here}]
    In calculus, \(\int_{a}^{b} f(x) \ dx\) can be computed as
    \[\int_{a}^{b} f(x) \ dx = F(b) - F(a),\]
    where \(F(x)\) is an arbitrary anti-derivative of \(f(x)\) on \([a,b]\), \textit{i.e.}, \(F'(x) = f(x)\) for all \(x\) in \([a,b]\). This formula for \(\int_{a}^{b} f(x) \ dx\) involves a choice of anti-derivative for \(f(x)\), but the formula does \textit{not} depend on the choice: every anti-derivative \(G(x)\) of \(f(x)\) on \([a,b]\) differs from \(F(x)\) by a constant, say \(G(x) = F(x) + C\) for all \(x\) in \([a,b]\), and changing the anti-derivative \(G(x)\) does not change the difference of its values at the endpoints:
    \[G(b) - G(a) = (F(b) + C) - (F(a) + C) = F(b) - F(a).\]
    So the difference of the values of an anti-derivative of \(f(x)\) at \(x=a\) and \(x=b\) is independent of the choice of anti-derivative of \(f(x)\) on the interval \([a,b]\).\footnote[2]{This is why in physics, potential energy has no intrinsic meaning (the zero level of potential energy can be anywhere), but \textit{differences} in potential energy are physically meaningful.}

    In contrast, the ``rule'' \(F(b)+F(a)\) depends on the choice of anti-derivative of \(f(x)\), since
    \[G(b)+G(a) = (F(b)+ C) + (F(a) + C) = F(b) + F(a) + 2C,\]
    which is a new value if \(C \neq 0\). Taking differences in an anti-derivative cancels the effect of the undetermined additive constant, so the expression \(F(b)-F(a)\) is a well-defined value based on the original input function \(f(x)\) and the interval \([a,b]\).
\end{exmp}

\begin{note}
    \lipsum[][1-9]
\end{note}

\begin{rmrk}
    \lipsum[][1-9]
\end{rmrk}

\begin{lemm}
    \lipsum[][1-4]
\end{lemm}

\begin{proof}
    \lipsum[1]
\end{proof}

\begin{thrm}[\ind{Thrm Ipsum}]
    \lipsum[][1-4]
\end{thrm}

\begin{proof}
    \lipsum[1]
\end{proof}

\begin{coro}
    \lipsum[][1-4]
\end{coro}

%%%%%%%%%%%%%%%%%%%%%%%
\subsection{Subsection}

\lipsum[][1-9]

\begin{defn}[\ind{Defn Ipsum}]
    \lipsum[][1] \(\sum_{i=1}^{n} n = \frac{n(n+1)}{2}\) \lipsum[][2-3]
    \[e^{i\pi}+1 = 0\]
    \lipsum[][4] \(\int_{a}^{b} x^2 \ dx = \frac{x^3}{3} + C\) \lipsum[][5-6]
\end{defn}

\begin{nota}
    \lipsum[][1-3]
    \[\sum_{i=1}^{n} n = \frac{n(n+1)}{2}\]
    \lipsum[][4-5]
    \[\int x \ dx = \frac{x^2}{2} + C\]
    \lipsum[][6-7]
\end{nota}

\begin{defn}
    %Given any natural number \(n\):
    \[\sum_{i=1}^{n} n = \frac{n(n+1)}{2}\]
    \lipsum[][3-5]
\end{defn}

\begin{rmrk}
    No environment should ever start with inline- or (especially not) display-mode math. Not only is that \href{https://kconrad.math.uconn.edu/blurbs/proofs/writingtips.pdf}{bad writing practice} but, in the case of starting with display-mode math such as above, blank vertical space will be left before the display-mode math where text is expected to be.

    If you don't know what to write, just state some context about the equation or expression with which you intended to start, e.g. `Given any natural number \(n\):'.
\end{rmrk}

\begin{exmp}[This is transcribed from Example 1.3 \href{https://kconrad.math.uconn.edu/blurbs/proofs/welldefined.pdf}{here}]
    In calculus, \(\int_{a}^{b} f(x) \ dx\) can be computed as
    \[\int_{a}^{b} f(x) \ dx = F(b) - F(a),\]
    where \(F(x)\) is an arbitrary anti-derivative of \(f(x)\) on \([a,b]\), \textit{i.e.}, \(F'(x) = f(x)\) for all \(x\) in \([a,b]\). This formula for \(\int_{a}^{b} f(x) \ dx\) involves a choice of anti-derivative for \(f(x)\), but the formula does \textit{not} depend on the choice: every anti-derivative \(G(x)\) of \(f(x)\) on \([a,b]\) differs from \(F(x)\) by a constant, say \(G(x) = F(x) + C\) for all \(x\) in \([a,b]\), and changing the anti-derivative \(G(x)\) does not change the difference of its values at the endpoints:
    \[G(b) - G(a) = (F(b) + C) - (F(a) + C) = F(b) - F(a).\]
    So the difference of the values of an anti-derivative of \(f(x)\) at \(x=a\) and \(x=b\) is independent of the choice of anti-derivative of \(f(x)\) on the interval \([a,b]\).\footnote[2]{This is why in physics, potential energy has no intrinsic meaning (the zero level of potential energy can be anywhere), but \textit{differences} in potential energy are physically meaningful.}

    In contrast, the ``rule'' \(F(b)+F(a)\) depends on the choice of anti-derivative of \(f(x)\), since
    \[G(b)+G(a) = (F(b)+ C) + (F(a) + C) = F(b) + F(a) + 2C,\]
    which is a new value if \(C \neq 0\). Taking differences in an anti-derivative cancels the effect of the undetermined additive constant, so the expression \(F(b)-F(a)\) is a well-defined value based on the original input function \(f(x)\) and the interval \([a,b]\).
\end{exmp}

\begin{note}
    \lipsum[][1-9]
\end{note}

\begin{rmrk}
    \lipsum[][1-9]
\end{rmrk}

\begin{lemm}
    \lipsum[][1-4]
\end{lemm}

\begin{proof}
    \lipsum[1]
\end{proof}

\begin{thrm}[\ind{Thrm Ipsum}]
    \lipsum[][1-4]
\end{thrm}

\begin{proof}
    \lipsum[1]
\end{proof}

\begin{coro}
    \lipsum[][1-4]
\end{coro}

\begin{exrc}
    \lipsum[1]
\end{exrc}

\begin{soln}
    \lipsum[1]
\end{soln}

%%%%%%%%%%%%%%%%%%%%%%%%%%%%%
\subsubsection{Subsubsection}

\begin{defn}[\ind{Defn Ipsum}]
    \lipsum[][1] \(\sum_{i=1}^{n} n = \frac{n(n+1)}{2}\) \lipsum[][2-3]
    \[e^{i\pi}+1 = 0\]
    \lipsum[][4] \(\int_{a}^{b} x^2 \ dx = \frac{x^3}{3} + C\) \lipsum[][5-6]
\end{defn}

\begin{nota}
    \lipsum[][1-3]
    \[\sum_{i=1}^{n} n = \frac{n(n+1)}{2}\]
    \lipsum[][4-5]
    \[\int x \ dx = \frac{x^2}{2} + C\]
    \lipsum[][6-7]
\end{nota}

\begin{defn}
    %Given any natural number \(n\):
    \[\sum_{i=1}^{n} n = \frac{n(n+1)}{2}\]
    \lipsum[][3-5]
\end{defn}

\begin{rmrk}
    No environment should ever start with inline- or (especially not) display-mode math. Not only is that \href{https://kconrad.math.uconn.edu/blurbs/proofs/writingtips.pdf}{bad writing practice} but, in the case of starting with display-mode math such as above, blank vertical space will be left before the display-mode math where text is expected to be.

    If you don't know what to write, just state some context about the equation or expression with which you intended to start, e.g. `Given any natural number \(n\):'.
\end{rmrk}

\begin{exmp}[This is transcribed from Example 1.3 \href{https://kconrad.math.uconn.edu/blurbs/proofs/welldefined.pdf}{here}]
    In calculus, \(\int_{a}^{b} f(x) \ dx\) can be computed as
    \[\int_{a}^{b} f(x) \ dx = F(b) - F(a),\]
    where \(F(x)\) is an arbitrary anti-derivative of \(f(x)\) on \([a,b]\), \textit{i.e.}, \(F'(x) = f(x)\) for all \(x\) in \([a,b]\). This formula for \(\int_{a}^{b} f(x) \ dx\) involves a choice of anti-derivative for \(f(x)\), but the formula does \textit{not} depend on the choice: every anti-derivative \(G(x)\) of \(f(x)\) on \([a,b]\) differs from \(F(x)\) by a constant, say \(G(x) = F(x) + C\) for all \(x\) in \([a,b]\), and changing the anti-derivative \(G(x)\) does not change the difference of its values at the endpoints:
    \[G(b) - G(a) = (F(b) + C) - (F(a) + C) = F(b) - F(a).\]
    So the difference of the values of an anti-derivative of \(f(x)\) at \(x=a\) and \(x=b\) is independent of the choice of anti-derivative of \(f(x)\) on the interval \([a,b]\).\footnote[2]{This is why in physics, potential energy has no intrinsic meaning (the zero level of potential energy can be anywhere), but \textit{differences} in potential energy are physically meaningful.}

    In contrast, the ``rule'' \(F(b)+F(a)\) depends on the choice of anti-derivative of \(f(x)\), since
    \[G(b)+G(a) = (F(b)+ C) + (F(a) + C) = F(b) + F(a) + 2C,\]
    which is a new value if \(C \neq 0\). Taking differences in an anti-derivative cancels the effect of the undetermined additive constant, so the expression \(F(b)-F(a)\) is a well-defined value based on the original input function \(f(x)\) and the interval \([a,b]\).
\end{exmp}

\begin{note}
    \lipsum[][1-9]
\end{note}

\begin{rmrk}
    \lipsum[][1-9]
\end{rmrk}

\begin{lemm}
    \lipsum[][1-4]
\end{lemm}

\begin{proof}
    \lipsum[1]
\end{proof}

\begin{thrm}[\ind{Thrm Ipsum}]
    \lipsum[][1-4]
\end{thrm}

\begin{proof}
    \lipsum[1]
\end{proof}

\begin{coro}
    \lipsum[][1-4]
\end{coro}

%%%%%%%%%%%%%%%%%%%%%%%%%%%%%
\subsubsection{Subsubsection}

\begin{defn}[\ind{Defn Ipsum}]
    \lipsum[][1] \(\sum_{i=1}^{n} n = \frac{n(n+1)}{2}\) \lipsum[][2-3]
    \[e^{i\pi}+1 = 0\]
    \lipsum[][4] \(\int_{a}^{b} x^2 \ dx = \frac{x^3}{3} + C\) \lipsum[][5-6]
\end{defn}

\begin{nota}
    \lipsum[][1-3]
    \[\sum_{i=1}^{n} n = \frac{n(n+1)}{2}\]
    \lipsum[][4-5]
    \[\int x \ dx = \frac{x^2}{2} + C\]
    \lipsum[][6-7]
\end{nota}

\begin{defn}
    %Given any natural number \(n\):
    \[\sum_{i=1}^{n} n = \frac{n(n+1)}{2}\]
    \lipsum[][3-5]
\end{defn}

\begin{rmrk}
    No environment should ever start with inline- or (especially not) display-mode math. Not only is that \href{https://kconrad.math.uconn.edu/blurbs/proofs/writingtips.pdf}{bad writing practice} but, in the case of starting with display-mode math such as above, blank vertical space will be left before the display-mode math where text is expected to be.

    If you don't know what to write, just state some context about the equation or expression with which you intended to start, e.g. `Given any natural number \(n\):'.
\end{rmrk}

\begin{exmp}[This is transcribed from Example 1.3 \href{https://kconrad.math.uconn.edu/blurbs/proofs/welldefined.pdf}{here}]
    In calculus, \(\int_{a}^{b} f(x) \ dx\) can be computed as
    \[\int_{a}^{b} f(x) \ dx = F(b) - F(a),\]
    where \(F(x)\) is an arbitrary anti-derivative of \(f(x)\) on \([a,b]\), \textit{i.e.}, \(F'(x) = f(x)\) for all \(x\) in \([a,b]\). This formula for \(\int_{a}^{b} f(x) \ dx\) involves a choice of anti-derivative for \(f(x)\), but the formula does \textit{not} depend on the choice: every anti-derivative \(G(x)\) of \(f(x)\) on \([a,b]\) differs from \(F(x)\) by a constant, say \(G(x) = F(x) + C\) for all \(x\) in \([a,b]\), and changing the anti-derivative \(G(x)\) does not change the difference of its values at the endpoints:
    \[G(b) - G(a) = (F(b) + C) - (F(a) + C) = F(b) - F(a).\]
    So the difference of the values of an anti-derivative of \(f(x)\) at \(x=a\) and \(x=b\) is independent of the choice of anti-derivative of \(f(x)\) on the interval \([a,b]\).\footnote[2]{This is why in physics, potential energy has no intrinsic meaning (the zero level of potential energy can be anywhere), but \textit{differences} in potential energy are physically meaningful.}

    In contrast, the ``rule'' \(F(b)+F(a)\) depends on the choice of anti-derivative of \(f(x)\), since
    \[G(b)+G(a) = (F(b)+ C) + (F(a) + C) = F(b) + F(a) + 2C,\]
    which is a new value if \(C \neq 0\). Taking differences in an anti-derivative cancels the effect of the undetermined additive constant, so the expression \(F(b)-F(a)\) is a well-defined value based on the original input function \(f(x)\) and the interval \([a,b]\).
\end{exmp}

\begin{note}
    \lipsum[][1-9]
\end{note}

\begin{rmrk}
    \lipsum[][1-9]
\end{rmrk}

\begin{lemm}
    \lipsum[][1-4]
\end{lemm}

\begin{proof}
    \lipsum[1]
\end{proof}

\begin{thrm}[\ind{Thrm Ipsum}]
    \lipsum[][1-4]
\end{thrm}

\begin{proof}
    \lipsum[1]
\end{proof}

\begin{coro}
    \lipsum[][1-4]
\end{coro}

%%%%%%%%%%%%%%%%%%%%%%%%%%%%%
\subsubsection{Subsubsection}

\begin{defn}[\ind{Defn Ipsum}]
    \lipsum[][1] \(\sum_{i=1}^{n} n = \frac{n(n+1)}{2}\) \lipsum[][2-3]
    \[e^{i\pi}+1 = 0\]
    \lipsum[][4] \(\int_{a}^{b} x^2 \ dx = \frac{x^3}{3} + C\) \lipsum[][5-6]
\end{defn}

\begin{nota}
    \lipsum[][1-3]
    \[\sum_{i=1}^{n} n = \frac{n(n+1)}{2}\]
    \lipsum[][4-5]
    \[\int x \ dx = \frac{x^2}{2} + C\]
    \lipsum[][6-7]
\end{nota}

\begin{defn}
    %Given any natural number \(n\):
    \[\sum_{i=1}^{n} n = \frac{n(n+1)}{2}\]
    \lipsum[][3-5]
\end{defn}

\begin{rmrk}
    No environment should ever start with inline- or (especially not) display-mode math. Not only is that \href{https://kconrad.math.uconn.edu/blurbs/proofs/writingtips.pdf}{bad writing practice} but, in the case of starting with display-mode math such as above, blank vertical space will be left before the display-mode math where text is expected to be.

    If you don't know what to write, just state some context about the equation or expression with which you intended to start, e.g. `Given any natural number \(n\):'.
\end{rmrk}

\begin{exmp}[This is transcribed from Example 1.3 \href{https://kconrad.math.uconn.edu/blurbs/proofs/welldefined.pdf}{here}]
    In calculus, \(\int_{a}^{b} f(x) \ dx\) can be computed as
    \[\int_{a}^{b} f(x) \ dx = F(b) - F(a),\]
    where \(F(x)\) is an arbitrary anti-derivative of \(f(x)\) on \([a,b]\), \textit{i.e.}, \(F'(x) = f(x)\) for all \(x\) in \([a,b]\). This formula for \(\int_{a}^{b} f(x) \ dx\) involves a choice of anti-derivative for \(f(x)\), but the formula does \textit{not} depend on the choice: every anti-derivative \(G(x)\) of \(f(x)\) on \([a,b]\) differs from \(F(x)\) by a constant, say \(G(x) = F(x) + C\) for all \(x\) in \([a,b]\), and changing the anti-derivative \(G(x)\) does not change the difference of its values at the endpoints:
    \[G(b) - G(a) = (F(b) + C) - (F(a) + C) = F(b) - F(a).\]
    So the difference of the values of an anti-derivative of \(f(x)\) at \(x=a\) and \(x=b\) is independent of the choice of anti-derivative of \(f(x)\) on the interval \([a,b]\).\footnote[2]{This is why in physics, potential energy has no intrinsic meaning (the zero level of potential energy can be anywhere), but \textit{differences} in potential energy are physically meaningful.}

    In contrast, the ``rule'' \(F(b)+F(a)\) depends on the choice of anti-derivative of \(f(x)\), since
    \[G(b)+G(a) = (F(b)+ C) + (F(a) + C) = F(b) + F(a) + 2C,\]
    which is a new value if \(C \neq 0\). Taking differences in an anti-derivative cancels the effect of the undetermined additive constant, so the expression \(F(b)-F(a)\) is a well-defined value based on the original input function \(f(x)\) and the interval \([a,b]\).
\end{exmp}

\begin{note}
    \lipsum[][1-9]
\end{note}

\begin{rmrk}
    \lipsum[][1-9]
\end{rmrk}

\begin{lemm}
    \lipsum[][1-4]
\end{lemm}

\begin{proof}
    \lipsum[1]
\end{proof}

\begin{thrm}[\ind{Thrm Ipsum}]
    \lipsum[][1-4]
\end{thrm}

\begin{proof}
    \lipsum[1]
\end{proof}

\begin{coro}
    \lipsum[][1-4]
\end{coro}

\section{Section}

\lipsum[1-2]

%%%%%%%%%%%%%%%%%%%%%%%
\subsection{Subsection}

\lipsum[][1-9]

\begin{defn}[\ind{Defn Ipsum}]
    \lipsum[][1] \(\sum_{i=1}^{n} n = \frac{n(n+1)}{2}\) \lipsum[][2-3]
    \[e^{i\pi}+1 = 0\]
    \lipsum[][4] \(\int_{a}^{b} x^2 \ dx = \frac{x^3}{3} + C\) \lipsum[][5-6]
\end{defn}

\begin{nota}
    \lipsum[][1-3]
    \[\sum_{i=1}^{n} n = \frac{n(n+1)}{2}\]
    \lipsum[][4-5]
    \[\int x \ dx = \frac{x^2}{2} + C\]
    \lipsum[][6-7]
\end{nota}

\begin{defn}
    %Given any natural number \(n\):
    \[\sum_{i=1}^{n} n = \frac{n(n+1)}{2}\]
    \lipsum[][3-5]
\end{defn}

\begin{rmrk}
    No environment should ever start with inline- or (especially not) display-mode math. Not only is that \href{https://kconrad.math.uconn.edu/blurbs/proofs/writingtips.pdf}{bad writing practice} but, in the case of starting with display-mode math such as above, blank vertical space will be left before the display-mode math where text is expected to be.

    If you don't know what to write, just state some context about the equation or expression with which you intended to start, e.g. `Given any natural number \(n\):'.
\end{rmrk}

\begin{exmp}[This is transcribed from Example 1.3 \href{https://kconrad.math.uconn.edu/blurbs/proofs/welldefined.pdf}{here}]
    In calculus, \(\int_{a}^{b} f(x) \ dx\) can be computed as
    \[\int_{a}^{b} f(x) \ dx = F(b) - F(a),\]
    where \(F(x)\) is an arbitrary anti-derivative of \(f(x)\) on \([a,b]\), \textit{i.e.}, \(F'(x) = f(x)\) for all \(x\) in \([a,b]\). This formula for \(\int_{a}^{b} f(x) \ dx\) involves a choice of anti-derivative for \(f(x)\), but the formula does \textit{not} depend on the choice: every anti-derivative \(G(x)\) of \(f(x)\) on \([a,b]\) differs from \(F(x)\) by a constant, say \(G(x) = F(x) + C\) for all \(x\) in \([a,b]\), and changing the anti-derivative \(G(x)\) does not change the difference of its values at the endpoints:
    \[G(b) - G(a) = (F(b) + C) - (F(a) + C) = F(b) - F(a).\]
    So the difference of the values of an anti-derivative of \(f(x)\) at \(x=a\) and \(x=b\) is independent of the choice of anti-derivative of \(f(x)\) on the interval \([a,b]\).\footnote[2]{This is why in physics, potential energy has no intrinsic meaning (the zero level of potential energy can be anywhere), but \textit{differences} in potential energy are physically meaningful.}

    In contrast, the ``rule'' \(F(b)+F(a)\) depends on the choice of anti-derivative of \(f(x)\), since
    \[G(b)+G(a) = (F(b)+ C) + (F(a) + C) = F(b) + F(a) + 2C,\]
    which is a new value if \(C \neq 0\). Taking differences in an anti-derivative cancels the effect of the undetermined additive constant, so the expression \(F(b)-F(a)\) is a well-defined value based on the original input function \(f(x)\) and the interval \([a,b]\).
\end{exmp}

\begin{note}
    \lipsum[][1-9]
\end{note}

\begin{rmrk}
    \lipsum[][1-9]
\end{rmrk}

\begin{lemm}
    \lipsum[][1-4]
\end{lemm}

\begin{proof}
    \lipsum[1]
\end{proof}

\begin{thrm}[\ind{Thrm Ipsum}]
    \lipsum[][1-4]
\end{thrm}

\begin{proof}
    \lipsum[1]
\end{proof}

\begin{coro}
    \lipsum[][1-4]
\end{coro}

\begin{exrc}
    \lipsum[1]
\end{exrc}

\begin{soln}
    \lipsum[1]
\end{soln}

%%%%%%%%%%%%%%%%%%%%%%%%%%%%%
\subsubsection{Subsubsection}

\begin{defn}[\ind{Defn Ipsum}]
    \lipsum[][1] \(\sum_{i=1}^{n} n = \frac{n(n+1)}{2}\) \lipsum[][2-3]
    \[e^{i\pi}+1 = 0\]
    \lipsum[][4] \(\int_{a}^{b} x^2 \ dx = \frac{x^3}{3} + C\) \lipsum[][5-6]
\end{defn}

\begin{nota}
    \lipsum[][1-3]
    \[\sum_{i=1}^{n} n = \frac{n(n+1)}{2}\]
    \lipsum[][4-5]
    \[\int x \ dx = \frac{x^2}{2} + C\]
    \lipsum[][6-7]
\end{nota}

\begin{defn}
    %Given any natural number \(n\):
    \[\sum_{i=1}^{n} n = \frac{n(n+1)}{2}\]
    \lipsum[][3-5]
\end{defn}

\begin{rmrk}
    No environment should ever start with inline- or (especially not) display-mode math. Not only is that \href{https://kconrad.math.uconn.edu/blurbs/proofs/writingtips.pdf}{bad writing practice} but, in the case of starting with display-mode math such as above, blank vertical space will be left before the display-mode math where text is expected to be.

    If you don't know what to write, just state some context about the equation or expression with which you intended to start, e.g. `Given any natural number \(n\):'.
\end{rmrk}

\begin{exmp}[This is transcribed from Example 1.3 \href{https://kconrad.math.uconn.edu/blurbs/proofs/welldefined.pdf}{here}]
    In calculus, \(\int_{a}^{b} f(x) \ dx\) can be computed as
    \[\int_{a}^{b} f(x) \ dx = F(b) - F(a),\]
    where \(F(x)\) is an arbitrary anti-derivative of \(f(x)\) on \([a,b]\), \textit{i.e.}, \(F'(x) = f(x)\) for all \(x\) in \([a,b]\). This formula for \(\int_{a}^{b} f(x) \ dx\) involves a choice of anti-derivative for \(f(x)\), but the formula does \textit{not} depend on the choice: every anti-derivative \(G(x)\) of \(f(x)\) on \([a,b]\) differs from \(F(x)\) by a constant, say \(G(x) = F(x) + C\) for all \(x\) in \([a,b]\), and changing the anti-derivative \(G(x)\) does not change the difference of its values at the endpoints:
    \[G(b) - G(a) = (F(b) + C) - (F(a) + C) = F(b) - F(a).\]
    So the difference of the values of an anti-derivative of \(f(x)\) at \(x=a\) and \(x=b\) is independent of the choice of anti-derivative of \(f(x)\) on the interval \([a,b]\).\footnote[2]{This is why in physics, potential energy has no intrinsic meaning (the zero level of potential energy can be anywhere), but \textit{differences} in potential energy are physically meaningful.}

    In contrast, the ``rule'' \(F(b)+F(a)\) depends on the choice of anti-derivative of \(f(x)\), since
    \[G(b)+G(a) = (F(b)+ C) + (F(a) + C) = F(b) + F(a) + 2C,\]
    which is a new value if \(C \neq 0\). Taking differences in an anti-derivative cancels the effect of the undetermined additive constant, so the expression \(F(b)-F(a)\) is a well-defined value based on the original input function \(f(x)\) and the interval \([a,b]\).
\end{exmp}

\begin{note}
    \lipsum[][1-9]
\end{note}

\begin{rmrk}
    \lipsum[][1-9]
\end{rmrk}

\begin{lemm}
    \lipsum[][1-4]
\end{lemm}

\begin{proof}
    \lipsum[1]
\end{proof}

\begin{thrm}[\ind{Thrm Ipsum}]
    \lipsum[][1-4]
\end{thrm}

\begin{proof}
    \lipsum[1]
\end{proof}

\begin{coro}
    \lipsum[][1-4]
\end{coro}

%%%%%%%%%%%%%%%%%%%%%%%%%%%%%
\subsubsection{Subsubsection}

\begin{defn}[\ind{Defn Ipsum}]
    \lipsum[][1] \(\sum_{i=1}^{n} n = \frac{n(n+1)}{2}\) \lipsum[][2-3]
    \[e^{i\pi}+1 = 0\]
    \lipsum[][4] \(\int_{a}^{b} x^2 \ dx = \frac{x^3}{3} + C\) \lipsum[][5-6]
\end{defn}

\begin{nota}
    \lipsum[][1-3]
    \[\sum_{i=1}^{n} n = \frac{n(n+1)}{2}\]
    \lipsum[][4-5]
    \[\int x \ dx = \frac{x^2}{2} + C\]
    \lipsum[][6-7]
\end{nota}

\begin{defn}
    %Given any natural number \(n\):
    \[\sum_{i=1}^{n} n = \frac{n(n+1)}{2}\]
    \lipsum[][3-5]
\end{defn}

\begin{rmrk}
    No environment should ever start with inline- or (especially not) display-mode math. Not only is that \href{https://kconrad.math.uconn.edu/blurbs/proofs/writingtips.pdf}{bad writing practice} but, in the case of starting with display-mode math such as above, blank vertical space will be left before the display-mode math where text is expected to be.

    If you don't know what to write, just state some context about the equation or expression with which you intended to start, e.g. `Given any natural number \(n\):'.
\end{rmrk}

\begin{exmp}[This is transcribed from Example 1.3 \href{https://kconrad.math.uconn.edu/blurbs/proofs/welldefined.pdf}{here}]
    In calculus, \(\int_{a}^{b} f(x) \ dx\) can be computed as
    \[\int_{a}^{b} f(x) \ dx = F(b) - F(a),\]
    where \(F(x)\) is an arbitrary anti-derivative of \(f(x)\) on \([a,b]\), \textit{i.e.}, \(F'(x) = f(x)\) for all \(x\) in \([a,b]\). This formula for \(\int_{a}^{b} f(x) \ dx\) involves a choice of anti-derivative for \(f(x)\), but the formula does \textit{not} depend on the choice: every anti-derivative \(G(x)\) of \(f(x)\) on \([a,b]\) differs from \(F(x)\) by a constant, say \(G(x) = F(x) + C\) for all \(x\) in \([a,b]\), and changing the anti-derivative \(G(x)\) does not change the difference of its values at the endpoints:
    \[G(b) - G(a) = (F(b) + C) - (F(a) + C) = F(b) - F(a).\]
    So the difference of the values of an anti-derivative of \(f(x)\) at \(x=a\) and \(x=b\) is independent of the choice of anti-derivative of \(f(x)\) on the interval \([a,b]\).\footnote[2]{This is why in physics, potential energy has no intrinsic meaning (the zero level of potential energy can be anywhere), but \textit{differences} in potential energy are physically meaningful.}

    In contrast, the ``rule'' \(F(b)+F(a)\) depends on the choice of anti-derivative of \(f(x)\), since
    \[G(b)+G(a) = (F(b)+ C) + (F(a) + C) = F(b) + F(a) + 2C,\]
    which is a new value if \(C \neq 0\). Taking differences in an anti-derivative cancels the effect of the undetermined additive constant, so the expression \(F(b)-F(a)\) is a well-defined value based on the original input function \(f(x)\) and the interval \([a,b]\).
\end{exmp}

\begin{note}
    \lipsum[][1-9]
\end{note}

\begin{rmrk}
    \lipsum[][1-9]
\end{rmrk}

\begin{lemm}
    \lipsum[][1-4]
\end{lemm}

\begin{proof}
    \lipsum[1]
\end{proof}

\begin{thrm}[\ind{Thrm Ipsum}]
    \lipsum[][1-4]
\end{thrm}

\begin{proof}
    \lipsum[1]
\end{proof}

\begin{coro}
    \lipsum[][1-4]
\end{coro}

%%%%%%%%%%%%%%%%%%%%%%%%%%%%%
\subsubsection{Subsubsection}

\begin{defn}[\ind{Defn Ipsum}]
    \lipsum[][1] \(\sum_{i=1}^{n} n = \frac{n(n+1)}{2}\) \lipsum[][2-3]
    \[e^{i\pi}+1 = 0\]
    \lipsum[][4] \(\int_{a}^{b} x^2 \ dx = \frac{x^3}{3} + C\) \lipsum[][5-6]
\end{defn}

\begin{nota}
    \lipsum[][1-3]
    \[\sum_{i=1}^{n} n = \frac{n(n+1)}{2}\]
    \lipsum[][4-5]
    \[\int x \ dx = \frac{x^2}{2} + C\]
    \lipsum[][6-7]
\end{nota}

\begin{defn}
    %Given any natural number \(n\):
    \[\sum_{i=1}^{n} n = \frac{n(n+1)}{2}\]
    \lipsum[][3-5]
\end{defn}

\begin{rmrk}
    No environment should ever start with inline- or (especially not) display-mode math. Not only is that \href{https://kconrad.math.uconn.edu/blurbs/proofs/writingtips.pdf}{bad writing practice} but, in the case of starting with display-mode math such as above, blank vertical space will be left before the display-mode math where text is expected to be.

    If you don't know what to write, just state some context about the equation or expression with which you intended to start, e.g. `Given any natural number \(n\):'.
\end{rmrk}

\begin{exmp}[This is transcribed from Example 1.3 \href{https://kconrad.math.uconn.edu/blurbs/proofs/welldefined.pdf}{here}]
    In calculus, \(\int_{a}^{b} f(x) \ dx\) can be computed as
    \[\int_{a}^{b} f(x) \ dx = F(b) - F(a),\]
    where \(F(x)\) is an arbitrary anti-derivative of \(f(x)\) on \([a,b]\), \textit{i.e.}, \(F'(x) = f(x)\) for all \(x\) in \([a,b]\). This formula for \(\int_{a}^{b} f(x) \ dx\) involves a choice of anti-derivative for \(f(x)\), but the formula does \textit{not} depend on the choice: every anti-derivative \(G(x)\) of \(f(x)\) on \([a,b]\) differs from \(F(x)\) by a constant, say \(G(x) = F(x) + C\) for all \(x\) in \([a,b]\), and changing the anti-derivative \(G(x)\) does not change the difference of its values at the endpoints:
    \[G(b) - G(a) = (F(b) + C) - (F(a) + C) = F(b) - F(a).\]
    So the difference of the values of an anti-derivative of \(f(x)\) at \(x=a\) and \(x=b\) is independent of the choice of anti-derivative of \(f(x)\) on the interval \([a,b]\).\footnote[2]{This is why in physics, potential energy has no intrinsic meaning (the zero level of potential energy can be anywhere), but \textit{differences} in potential energy are physically meaningful.}

    In contrast, the ``rule'' \(F(b)+F(a)\) depends on the choice of anti-derivative of \(f(x)\), since
    \[G(b)+G(a) = (F(b)+ C) + (F(a) + C) = F(b) + F(a) + 2C,\]
    which is a new value if \(C \neq 0\). Taking differences in an anti-derivative cancels the effect of the undetermined additive constant, so the expression \(F(b)-F(a)\) is a well-defined value based on the original input function \(f(x)\) and the interval \([a,b]\).
\end{exmp}

\begin{note}
    \lipsum[][1-9]
\end{note}

\begin{rmrk}
    \lipsum[][1-9]
\end{rmrk}

\begin{lemm}
    \lipsum[][1-4]
\end{lemm}

\begin{proof}
    \lipsum[1]
\end{proof}

\begin{thrm}[\ind{Thrm Ipsum}]
    \lipsum[][1-4]
\end{thrm}

\begin{proof}
    \lipsum[1]
\end{proof}

\begin{coro}
    \lipsum[][1-4]
\end{coro}

%%%%%%%%%%%%%%%%%%%%%%%
\subsection{Subsection}

\lipsum[][1-9]

\begin{defn}[\ind{Defn Ipsum}]
    \lipsum[][1] \(\sum_{i=1}^{n} n = \frac{n(n+1)}{2}\) \lipsum[][2-3]
    \[e^{i\pi}+1 = 0\]
    \lipsum[][4] \(\int_{a}^{b} x^2 \ dx = \frac{x^3}{3} + C\) \lipsum[][5-6]
\end{defn}

\begin{nota}
    \lipsum[][1-3]
    \[\sum_{i=1}^{n} n = \frac{n(n+1)}{2}\]
    \lipsum[][4-5]
    \[\int x \ dx = \frac{x^2}{2} + C\]
    \lipsum[][6-7]
\end{nota}

\begin{defn}
    %Given any natural number \(n\):
    \[\sum_{i=1}^{n} n = \frac{n(n+1)}{2}\]
    \lipsum[][3-5]
\end{defn}

\begin{rmrk}
    No environment should ever start with inline- or (especially not) display-mode math. Not only is that \href{https://kconrad.math.uconn.edu/blurbs/proofs/writingtips.pdf}{bad writing practice} but, in the case of starting with display-mode math such as above, blank vertical space will be left before the display-mode math where text is expected to be.

    If you don't know what to write, just state some context about the equation or expression with which you intended to start, e.g. `Given any natural number \(n\):'.
\end{rmrk}

\begin{exmp}[This is transcribed from Example 1.3 \href{https://kconrad.math.uconn.edu/blurbs/proofs/welldefined.pdf}{here}]
    In calculus, \(\int_{a}^{b} f(x) \ dx\) can be computed as
    \[\int_{a}^{b} f(x) \ dx = F(b) - F(a),\]
    where \(F(x)\) is an arbitrary anti-derivative of \(f(x)\) on \([a,b]\), \textit{i.e.}, \(F'(x) = f(x)\) for all \(x\) in \([a,b]\). This formula for \(\int_{a}^{b} f(x) \ dx\) involves a choice of anti-derivative for \(f(x)\), but the formula does \textit{not} depend on the choice: every anti-derivative \(G(x)\) of \(f(x)\) on \([a,b]\) differs from \(F(x)\) by a constant, say \(G(x) = F(x) + C\) for all \(x\) in \([a,b]\), and changing the anti-derivative \(G(x)\) does not change the difference of its values at the endpoints:
    \[G(b) - G(a) = (F(b) + C) - (F(a) + C) = F(b) - F(a).\]
    So the difference of the values of an anti-derivative of \(f(x)\) at \(x=a\) and \(x=b\) is independent of the choice of anti-derivative of \(f(x)\) on the interval \([a,b]\).\footnote[2]{This is why in physics, potential energy has no intrinsic meaning (the zero level of potential energy can be anywhere), but \textit{differences} in potential energy are physically meaningful.}

    In contrast, the ``rule'' \(F(b)+F(a)\) depends on the choice of anti-derivative of \(f(x)\), since
    \[G(b)+G(a) = (F(b)+ C) + (F(a) + C) = F(b) + F(a) + 2C,\]
    which is a new value if \(C \neq 0\). Taking differences in an anti-derivative cancels the effect of the undetermined additive constant, so the expression \(F(b)-F(a)\) is a well-defined value based on the original input function \(f(x)\) and the interval \([a,b]\).
\end{exmp}

\begin{note}
    \lipsum[][1-9]
\end{note}

\begin{rmrk}
    \lipsum[][1-9]
\end{rmrk}

\begin{lemm}
    \lipsum[][1-4]
\end{lemm}

\begin{proof}
    \lipsum[1]
\end{proof}

\begin{thrm}[\ind{Thrm Ipsum}]
    \lipsum[][1-4]
\end{thrm}

\begin{proof}
    \lipsum[1]
\end{proof}

\begin{coro}
    \lipsum[][1-4]
\end{coro}

\begin{exrc}
    \lipsum[1]
\end{exrc}

\begin{soln}
    \lipsum[1]
\end{soln}

%%%%%%%%%%%%%%%%%%%%%%%%%%%%%
\subsubsection{Subsubsection}

\begin{defn}[\ind{Defn Ipsum}]
    \lipsum[][1] \(\sum_{i=1}^{n} n = \frac{n(n+1)}{2}\) \lipsum[][2-3]
    \[e^{i\pi}+1 = 0\]
    \lipsum[][4] \(\int_{a}^{b} x^2 \ dx = \frac{x^3}{3} + C\) \lipsum[][5-6]
\end{defn}

\begin{nota}
    \lipsum[][1-3]
    \[\sum_{i=1}^{n} n = \frac{n(n+1)}{2}\]
    \lipsum[][4-5]
    \[\int x \ dx = \frac{x^2}{2} + C\]
    \lipsum[][6-7]
\end{nota}

\begin{defn}
    %Given any natural number \(n\):
    \[\sum_{i=1}^{n} n = \frac{n(n+1)}{2}\]
    \lipsum[][3-5]
\end{defn}

\begin{rmrk}
    No environment should ever start with inline- or (especially not) display-mode math. Not only is that \href{https://kconrad.math.uconn.edu/blurbs/proofs/writingtips.pdf}{bad writing practice} but, in the case of starting with display-mode math such as above, blank vertical space will be left before the display-mode math where text is expected to be.

    If you don't know what to write, just state some context about the equation or expression with which you intended to start, e.g. `Given any natural number \(n\):'.
\end{rmrk}

\begin{exmp}[This is transcribed from Example 1.3 \href{https://kconrad.math.uconn.edu/blurbs/proofs/welldefined.pdf}{here}]
    In calculus, \(\int_{a}^{b} f(x) \ dx\) can be computed as
    \[\int_{a}^{b} f(x) \ dx = F(b) - F(a),\]
    where \(F(x)\) is an arbitrary anti-derivative of \(f(x)\) on \([a,b]\), \textit{i.e.}, \(F'(x) = f(x)\) for all \(x\) in \([a,b]\). This formula for \(\int_{a}^{b} f(x) \ dx\) involves a choice of anti-derivative for \(f(x)\), but the formula does \textit{not} depend on the choice: every anti-derivative \(G(x)\) of \(f(x)\) on \([a,b]\) differs from \(F(x)\) by a constant, say \(G(x) = F(x) + C\) for all \(x\) in \([a,b]\), and changing the anti-derivative \(G(x)\) does not change the difference of its values at the endpoints:
    \[G(b) - G(a) = (F(b) + C) - (F(a) + C) = F(b) - F(a).\]
    So the difference of the values of an anti-derivative of \(f(x)\) at \(x=a\) and \(x=b\) is independent of the choice of anti-derivative of \(f(x)\) on the interval \([a,b]\).\footnote[2]{This is why in physics, potential energy has no intrinsic meaning (the zero level of potential energy can be anywhere), but \textit{differences} in potential energy are physically meaningful.}

    In contrast, the ``rule'' \(F(b)+F(a)\) depends on the choice of anti-derivative of \(f(x)\), since
    \[G(b)+G(a) = (F(b)+ C) + (F(a) + C) = F(b) + F(a) + 2C,\]
    which is a new value if \(C \neq 0\). Taking differences in an anti-derivative cancels the effect of the undetermined additive constant, so the expression \(F(b)-F(a)\) is a well-defined value based on the original input function \(f(x)\) and the interval \([a,b]\).
\end{exmp}

\begin{note}
    \lipsum[][1-9]
\end{note}

\begin{rmrk}
    \lipsum[][1-9]
\end{rmrk}

\begin{lemm}
    \lipsum[][1-4]
\end{lemm}

\begin{proof}
    \lipsum[1]
\end{proof}

\begin{thrm}[\ind{Thrm Ipsum}]
    \lipsum[][1-4]
\end{thrm}

\begin{proof}
    \lipsum[1]
\end{proof}

\begin{coro}
    \lipsum[][1-4]
\end{coro}

%%%%%%%%%%%%%%%%%%%%%%%%%%%%%
\subsubsection{Subsubsection}

\begin{defn}[\ind{Defn Ipsum}]
    \lipsum[][1] \(\sum_{i=1}^{n} n = \frac{n(n+1)}{2}\) \lipsum[][2-3]
    \[e^{i\pi}+1 = 0\]
    \lipsum[][4] \(\int_{a}^{b} x^2 \ dx = \frac{x^3}{3} + C\) \lipsum[][5-6]
\end{defn}

\begin{nota}
    \lipsum[][1-3]
    \[\sum_{i=1}^{n} n = \frac{n(n+1)}{2}\]
    \lipsum[][4-5]
    \[\int x \ dx = \frac{x^2}{2} + C\]
    \lipsum[][6-7]
\end{nota}

\begin{defn}
    %Given any natural number \(n\):
    \[\sum_{i=1}^{n} n = \frac{n(n+1)}{2}\]
    \lipsum[][3-5]
\end{defn}

\begin{rmrk}
    No environment should ever start with inline- or (especially not) display-mode math. Not only is that \href{https://kconrad.math.uconn.edu/blurbs/proofs/writingtips.pdf}{bad writing practice} but, in the case of starting with display-mode math such as above, blank vertical space will be left before the display-mode math where text is expected to be.

    If you don't know what to write, just state some context about the equation or expression with which you intended to start, e.g. `Given any natural number \(n\):'.
\end{rmrk}

\begin{exmp}[This is transcribed from Example 1.3 \href{https://kconrad.math.uconn.edu/blurbs/proofs/welldefined.pdf}{here}]
    In calculus, \(\int_{a}^{b} f(x) \ dx\) can be computed as
    \[\int_{a}^{b} f(x) \ dx = F(b) - F(a),\]
    where \(F(x)\) is an arbitrary anti-derivative of \(f(x)\) on \([a,b]\), \textit{i.e.}, \(F'(x) = f(x)\) for all \(x\) in \([a,b]\). This formula for \(\int_{a}^{b} f(x) \ dx\) involves a choice of anti-derivative for \(f(x)\), but the formula does \textit{not} depend on the choice: every anti-derivative \(G(x)\) of \(f(x)\) on \([a,b]\) differs from \(F(x)\) by a constant, say \(G(x) = F(x) + C\) for all \(x\) in \([a,b]\), and changing the anti-derivative \(G(x)\) does not change the difference of its values at the endpoints:
    \[G(b) - G(a) = (F(b) + C) - (F(a) + C) = F(b) - F(a).\]
    So the difference of the values of an anti-derivative of \(f(x)\) at \(x=a\) and \(x=b\) is independent of the choice of anti-derivative of \(f(x)\) on the interval \([a,b]\).\footnote[2]{This is why in physics, potential energy has no intrinsic meaning (the zero level of potential energy can be anywhere), but \textit{differences} in potential energy are physically meaningful.}

    In contrast, the ``rule'' \(F(b)+F(a)\) depends on the choice of anti-derivative of \(f(x)\), since
    \[G(b)+G(a) = (F(b)+ C) + (F(a) + C) = F(b) + F(a) + 2C,\]
    which is a new value if \(C \neq 0\). Taking differences in an anti-derivative cancels the effect of the undetermined additive constant, so the expression \(F(b)-F(a)\) is a well-defined value based on the original input function \(f(x)\) and the interval \([a,b]\).
\end{exmp}

\begin{note}
    \lipsum[][1-9]
\end{note}

\begin{rmrk}
    \lipsum[][1-9]
\end{rmrk}

\begin{lemm}
    \lipsum[][1-4]
\end{lemm}

\begin{proof}
    \lipsum[1]
\end{proof}

\begin{thrm}[\ind{Thrm Ipsum}]
    \lipsum[][1-4]
\end{thrm}

\begin{proof}
    \lipsum[1]
\end{proof}

\begin{coro}
    \lipsum[][1-4]
\end{coro}

%%%%%%%%%%%%%%%%%%%%%%%%%%%%%
\subsubsection{Subsubsection}

\begin{defn}[\ind{Defn Ipsum}]
    \lipsum[][1] \(\sum_{i=1}^{n} n = \frac{n(n+1)}{2}\) \lipsum[][2-3]
    \[e^{i\pi}+1 = 0\]
    \lipsum[][4] \(\int_{a}^{b} x^2 \ dx = \frac{x^3}{3} + C\) \lipsum[][5-6]
\end{defn}

\begin{nota}
    \lipsum[][1-3]
    \[\sum_{i=1}^{n} n = \frac{n(n+1)}{2}\]
    \lipsum[][4-5]
    \[\int x \ dx = \frac{x^2}{2} + C\]
    \lipsum[][6-7]
\end{nota}

\begin{defn}
    %Given any natural number \(n\):
    \[\sum_{i=1}^{n} n = \frac{n(n+1)}{2}\]
    \lipsum[][3-5]
\end{defn}

\begin{rmrk}
    No environment should ever start with inline- or (especially not) display-mode math. Not only is that \href{https://kconrad.math.uconn.edu/blurbs/proofs/writingtips.pdf}{bad writing practice} but, in the case of starting with display-mode math such as above, blank vertical space will be left before the display-mode math where text is expected to be.

    If you don't know what to write, just state some context about the equation or expression with which you intended to start, e.g. `Given any natural number \(n\):'.
\end{rmrk}

\begin{exmp}[This is transcribed from Example 1.3 \href{https://kconrad.math.uconn.edu/blurbs/proofs/welldefined.pdf}{here}]
    In calculus, \(\int_{a}^{b} f(x) \ dx\) can be computed as
    \[\int_{a}^{b} f(x) \ dx = F(b) - F(a),\]
    where \(F(x)\) is an arbitrary anti-derivative of \(f(x)\) on \([a,b]\), \textit{i.e.}, \(F'(x) = f(x)\) for all \(x\) in \([a,b]\). This formula for \(\int_{a}^{b} f(x) \ dx\) involves a choice of anti-derivative for \(f(x)\), but the formula does \textit{not} depend on the choice: every anti-derivative \(G(x)\) of \(f(x)\) on \([a,b]\) differs from \(F(x)\) by a constant, say \(G(x) = F(x) + C\) for all \(x\) in \([a,b]\), and changing the anti-derivative \(G(x)\) does not change the difference of its values at the endpoints:
    \[G(b) - G(a) = (F(b) + C) - (F(a) + C) = F(b) - F(a).\]
    So the difference of the values of an anti-derivative of \(f(x)\) at \(x=a\) and \(x=b\) is independent of the choice of anti-derivative of \(f(x)\) on the interval \([a,b]\).\footnote[2]{This is why in physics, potential energy has no intrinsic meaning (the zero level of potential energy can be anywhere), but \textit{differences} in potential energy are physically meaningful.}

    In contrast, the ``rule'' \(F(b)+F(a)\) depends on the choice of anti-derivative of \(f(x)\), since
    \[G(b)+G(a) = (F(b)+ C) + (F(a) + C) = F(b) + F(a) + 2C,\]
    which is a new value if \(C \neq 0\). Taking differences in an anti-derivative cancels the effect of the undetermined additive constant, so the expression \(F(b)-F(a)\) is a well-defined value based on the original input function \(f(x)\) and the interval \([a,b]\).
\end{exmp}

\begin{note}
    \lipsum[][1-9]
\end{note}

\begin{rmrk}
    \lipsum[][1-9]
\end{rmrk}

\begin{lemm}
    \lipsum[][1-4]
\end{lemm}

\begin{proof}
    \lipsum[1]
\end{proof}

\begin{thrm}[\ind{Thrm Ipsum}]
    \lipsum[][1-4]
\end{thrm}

\begin{proof}
    \lipsum[1]
\end{proof}

\begin{coro}
    \lipsum[][1-4]
\end{coro}

%%%%%%%%%%%%%%%%%%%%%%%
\subsection{Subsection}

\lipsum[][1-9]

\begin{defn}[\ind{Defn Ipsum}]
    \lipsum[][1] \(\sum_{i=1}^{n} n = \frac{n(n+1)}{2}\) \lipsum[][2-3]
    \[e^{i\pi}+1 = 0\]
    \lipsum[][4] \(\int_{a}^{b} x^2 \ dx = \frac{x^3}{3} + C\) \lipsum[][5-6]
\end{defn}

\begin{nota}
    \lipsum[][1-3]
    \[\sum_{i=1}^{n} n = \frac{n(n+1)}{2}\]
    \lipsum[][4-5]
    \[\int x \ dx = \frac{x^2}{2} + C\]
    \lipsum[][6-7]
\end{nota}

\begin{defn}
    %Given any natural number \(n\):
    \[\sum_{i=1}^{n} n = \frac{n(n+1)}{2}\]
    \lipsum[][3-5]
\end{defn}

\begin{rmrk}
    No environment should ever start with inline- or (especially not) display-mode math. Not only is that \href{https://kconrad.math.uconn.edu/blurbs/proofs/writingtips.pdf}{bad writing practice} but, in the case of starting with display-mode math such as above, blank vertical space will be left before the display-mode math where text is expected to be.

    If you don't know what to write, just state some context about the equation or expression with which you intended to start, e.g. `Given any natural number \(n\):'.
\end{rmrk}

\begin{exmp}[This is transcribed from Example 1.3 \href{https://kconrad.math.uconn.edu/blurbs/proofs/welldefined.pdf}{here}]
    In calculus, \(\int_{a}^{b} f(x) \ dx\) can be computed as
    \[\int_{a}^{b} f(x) \ dx = F(b) - F(a),\]
    where \(F(x)\) is an arbitrary anti-derivative of \(f(x)\) on \([a,b]\), \textit{i.e.}, \(F'(x) = f(x)\) for all \(x\) in \([a,b]\). This formula for \(\int_{a}^{b} f(x) \ dx\) involves a choice of anti-derivative for \(f(x)\), but the formula does \textit{not} depend on the choice: every anti-derivative \(G(x)\) of \(f(x)\) on \([a,b]\) differs from \(F(x)\) by a constant, say \(G(x) = F(x) + C\) for all \(x\) in \([a,b]\), and changing the anti-derivative \(G(x)\) does not change the difference of its values at the endpoints:
    \[G(b) - G(a) = (F(b) + C) - (F(a) + C) = F(b) - F(a).\]
    So the difference of the values of an anti-derivative of \(f(x)\) at \(x=a\) and \(x=b\) is independent of the choice of anti-derivative of \(f(x)\) on the interval \([a,b]\).\footnote[2]{This is why in physics, potential energy has no intrinsic meaning (the zero level of potential energy can be anywhere), but \textit{differences} in potential energy are physically meaningful.}

    In contrast, the ``rule'' \(F(b)+F(a)\) depends on the choice of anti-derivative of \(f(x)\), since
    \[G(b)+G(a) = (F(b)+ C) + (F(a) + C) = F(b) + F(a) + 2C,\]
    which is a new value if \(C \neq 0\). Taking differences in an anti-derivative cancels the effect of the undetermined additive constant, so the expression \(F(b)-F(a)\) is a well-defined value based on the original input function \(f(x)\) and the interval \([a,b]\).
\end{exmp}

\begin{note}
    \lipsum[][1-9]
\end{note}

\begin{rmrk}
    \lipsum[][1-9]
\end{rmrk}

\begin{lemm}
    \lipsum[][1-4]
\end{lemm}

\begin{proof}
    \lipsum[1]
\end{proof}

\begin{thrm}[\ind{Thrm Ipsum}]
    \lipsum[][1-4]
\end{thrm}

\begin{proof}
    \lipsum[1]
\end{proof}

\begin{coro}
    \lipsum[][1-4]
\end{coro}

\begin{exrc}
    \lipsum[1]
\end{exrc}

\begin{soln}
    \lipsum[1]
\end{soln}

%%%%%%%%%%%%%%%%%%%%%%%%%%%%%
\subsubsection{Subsubsection}

\begin{defn}[\ind{Defn Ipsum}]
    \lipsum[][1] \(\sum_{i=1}^{n} n = \frac{n(n+1)}{2}\) \lipsum[][2-3]
    \[e^{i\pi}+1 = 0\]
    \lipsum[][4] \(\int_{a}^{b} x^2 \ dx = \frac{x^3}{3} + C\) \lipsum[][5-6]
\end{defn}

\begin{nota}
    \lipsum[][1-3]
    \[\sum_{i=1}^{n} n = \frac{n(n+1)}{2}\]
    \lipsum[][4-5]
    \[\int x \ dx = \frac{x^2}{2} + C\]
    \lipsum[][6-7]
\end{nota}

\begin{defn}
    %Given any natural number \(n\):
    \[\sum_{i=1}^{n} n = \frac{n(n+1)}{2}\]
    \lipsum[][3-5]
\end{defn}

\begin{rmrk}
    No environment should ever start with inline- or (especially not) display-mode math. Not only is that \href{https://kconrad.math.uconn.edu/blurbs/proofs/writingtips.pdf}{bad writing practice} but, in the case of starting with display-mode math such as above, blank vertical space will be left before the display-mode math where text is expected to be.

    If you don't know what to write, just state some context about the equation or expression with which you intended to start, e.g. `Given any natural number \(n\):'.
\end{rmrk}

\begin{exmp}[This is transcribed from Example 1.3 \href{https://kconrad.math.uconn.edu/blurbs/proofs/welldefined.pdf}{here}]
    In calculus, \(\int_{a}^{b} f(x) \ dx\) can be computed as
    \[\int_{a}^{b} f(x) \ dx = F(b) - F(a),\]
    where \(F(x)\) is an arbitrary anti-derivative of \(f(x)\) on \([a,b]\), \textit{i.e.}, \(F'(x) = f(x)\) for all \(x\) in \([a,b]\). This formula for \(\int_{a}^{b} f(x) \ dx\) involves a choice of anti-derivative for \(f(x)\), but the formula does \textit{not} depend on the choice: every anti-derivative \(G(x)\) of \(f(x)\) on \([a,b]\) differs from \(F(x)\) by a constant, say \(G(x) = F(x) + C\) for all \(x\) in \([a,b]\), and changing the anti-derivative \(G(x)\) does not change the difference of its values at the endpoints:
    \[G(b) - G(a) = (F(b) + C) - (F(a) + C) = F(b) - F(a).\]
    So the difference of the values of an anti-derivative of \(f(x)\) at \(x=a\) and \(x=b\) is independent of the choice of anti-derivative of \(f(x)\) on the interval \([a,b]\).\footnote[2]{This is why in physics, potential energy has no intrinsic meaning (the zero level of potential energy can be anywhere), but \textit{differences} in potential energy are physically meaningful.}

    In contrast, the ``rule'' \(F(b)+F(a)\) depends on the choice of anti-derivative of \(f(x)\), since
    \[G(b)+G(a) = (F(b)+ C) + (F(a) + C) = F(b) + F(a) + 2C,\]
    which is a new value if \(C \neq 0\). Taking differences in an anti-derivative cancels the effect of the undetermined additive constant, so the expression \(F(b)-F(a)\) is a well-defined value based on the original input function \(f(x)\) and the interval \([a,b]\).
\end{exmp}

\begin{note}
    \lipsum[][1-9]
\end{note}

\begin{rmrk}
    \lipsum[][1-9]
\end{rmrk}

\begin{lemm}
    \lipsum[][1-4]
\end{lemm}

\begin{proof}
    \lipsum[1]
\end{proof}

\begin{thrm}[\ind{Thrm Ipsum}]
    \lipsum[][1-4]
\end{thrm}

\begin{proof}
    \lipsum[1]
\end{proof}

\begin{coro}
    \lipsum[][1-4]
\end{coro}

%%%%%%%%%%%%%%%%%%%%%%%%%%%%%
\subsubsection{Subsubsection}

\begin{defn}[\ind{Defn Ipsum}]
    \lipsum[][1] \(\sum_{i=1}^{n} n = \frac{n(n+1)}{2}\) \lipsum[][2-3]
    \[e^{i\pi}+1 = 0\]
    \lipsum[][4] \(\int_{a}^{b} x^2 \ dx = \frac{x^3}{3} + C\) \lipsum[][5-6]
\end{defn}

\begin{nota}
    \lipsum[][1-3]
    \[\sum_{i=1}^{n} n = \frac{n(n+1)}{2}\]
    \lipsum[][4-5]
    \[\int x \ dx = \frac{x^2}{2} + C\]
    \lipsum[][6-7]
\end{nota}

\begin{defn}
    %Given any natural number \(n\):
    \[\sum_{i=1}^{n} n = \frac{n(n+1)}{2}\]
    \lipsum[][3-5]
\end{defn}

\begin{rmrk}
    No environment should ever start with inline- or (especially not) display-mode math. Not only is that \href{https://kconrad.math.uconn.edu/blurbs/proofs/writingtips.pdf}{bad writing practice} but, in the case of starting with display-mode math such as above, blank vertical space will be left before the display-mode math where text is expected to be.

    If you don't know what to write, just state some context about the equation or expression with which you intended to start, e.g. `Given any natural number \(n\):'.
\end{rmrk}

\begin{exmp}[This is transcribed from Example 1.3 \href{https://kconrad.math.uconn.edu/blurbs/proofs/welldefined.pdf}{here}]
    In calculus, \(\int_{a}^{b} f(x) \ dx\) can be computed as
    \[\int_{a}^{b} f(x) \ dx = F(b) - F(a),\]
    where \(F(x)\) is an arbitrary anti-derivative of \(f(x)\) on \([a,b]\), \textit{i.e.}, \(F'(x) = f(x)\) for all \(x\) in \([a,b]\). This formula for \(\int_{a}^{b} f(x) \ dx\) involves a choice of anti-derivative for \(f(x)\), but the formula does \textit{not} depend on the choice: every anti-derivative \(G(x)\) of \(f(x)\) on \([a,b]\) differs from \(F(x)\) by a constant, say \(G(x) = F(x) + C\) for all \(x\) in \([a,b]\), and changing the anti-derivative \(G(x)\) does not change the difference of its values at the endpoints:
    \[G(b) - G(a) = (F(b) + C) - (F(a) + C) = F(b) - F(a).\]
    So the difference of the values of an anti-derivative of \(f(x)\) at \(x=a\) and \(x=b\) is independent of the choice of anti-derivative of \(f(x)\) on the interval \([a,b]\).\footnote[2]{This is why in physics, potential energy has no intrinsic meaning (the zero level of potential energy can be anywhere), but \textit{differences} in potential energy are physically meaningful.}

    In contrast, the ``rule'' \(F(b)+F(a)\) depends on the choice of anti-derivative of \(f(x)\), since
    \[G(b)+G(a) = (F(b)+ C) + (F(a) + C) = F(b) + F(a) + 2C,\]
    which is a new value if \(C \neq 0\). Taking differences in an anti-derivative cancels the effect of the undetermined additive constant, so the expression \(F(b)-F(a)\) is a well-defined value based on the original input function \(f(x)\) and the interval \([a,b]\).
\end{exmp}

\begin{note}
    \lipsum[][1-9]
\end{note}

\begin{rmrk}
    \lipsum[][1-9]
\end{rmrk}

\begin{lemm}
    \lipsum[][1-4]
\end{lemm}

\begin{proof}
    \lipsum[1]
\end{proof}

\begin{thrm}[\ind{Thrm Ipsum}]
    \lipsum[][1-4]
\end{thrm}

\begin{proof}
    \lipsum[1]
\end{proof}

\begin{coro}
    \lipsum[][1-4]
\end{coro}

%%%%%%%%%%%%%%%%%%%%%%%%%%%%%
\subsubsection{Subsubsection}

\begin{defn}[\ind{Defn Ipsum}]
    \lipsum[][1] \(\sum_{i=1}^{n} n = \frac{n(n+1)}{2}\) \lipsum[][2-3]
    \[e^{i\pi}+1 = 0\]
    \lipsum[][4] \(\int_{a}^{b} x^2 \ dx = \frac{x^3}{3} + C\) \lipsum[][5-6]
\end{defn}

\begin{nota}
    \lipsum[][1-3]
    \[\sum_{i=1}^{n} n = \frac{n(n+1)}{2}\]
    \lipsum[][4-5]
    \[\int x \ dx = \frac{x^2}{2} + C\]
    \lipsum[][6-7]
\end{nota}

\begin{defn}
    %Given any natural number \(n\):
    \[\sum_{i=1}^{n} n = \frac{n(n+1)}{2}\]
    \lipsum[][3-5]
\end{defn}

\begin{rmrk}
    No environment should ever start with inline- or (especially not) display-mode math. Not only is that \href{https://kconrad.math.uconn.edu/blurbs/proofs/writingtips.pdf}{bad writing practice} but, in the case of starting with display-mode math such as above, blank vertical space will be left before the display-mode math where text is expected to be.

    If you don't know what to write, just state some context about the equation or expression with which you intended to start, e.g. `Given any natural number \(n\):'.
\end{rmrk}

\begin{exmp}[This is transcribed from Example 1.3 \href{https://kconrad.math.uconn.edu/blurbs/proofs/welldefined.pdf}{here}]
    In calculus, \(\int_{a}^{b} f(x) \ dx\) can be computed as
    \[\int_{a}^{b} f(x) \ dx = F(b) - F(a),\]
    where \(F(x)\) is an arbitrary anti-derivative of \(f(x)\) on \([a,b]\), \textit{i.e.}, \(F'(x) = f(x)\) for all \(x\) in \([a,b]\). This formula for \(\int_{a}^{b} f(x) \ dx\) involves a choice of anti-derivative for \(f(x)\), but the formula does \textit{not} depend on the choice: every anti-derivative \(G(x)\) of \(f(x)\) on \([a,b]\) differs from \(F(x)\) by a constant, say \(G(x) = F(x) + C\) for all \(x\) in \([a,b]\), and changing the anti-derivative \(G(x)\) does not change the difference of its values at the endpoints:
    \[G(b) - G(a) = (F(b) + C) - (F(a) + C) = F(b) - F(a).\]
    So the difference of the values of an anti-derivative of \(f(x)\) at \(x=a\) and \(x=b\) is independent of the choice of anti-derivative of \(f(x)\) on the interval \([a,b]\).\footnote[2]{This is why in physics, potential energy has no intrinsic meaning (the zero level of potential energy can be anywhere), but \textit{differences} in potential energy are physically meaningful.}

    In contrast, the ``rule'' \(F(b)+F(a)\) depends on the choice of anti-derivative of \(f(x)\), since
    \[G(b)+G(a) = (F(b)+ C) + (F(a) + C) = F(b) + F(a) + 2C,\]
    which is a new value if \(C \neq 0\). Taking differences in an anti-derivative cancels the effect of the undetermined additive constant, so the expression \(F(b)-F(a)\) is a well-defined value based on the original input function \(f(x)\) and the interval \([a,b]\).
\end{exmp}

\begin{note}
    \lipsum[][1-9]
\end{note}

\begin{rmrk}
    \lipsum[][1-9]
\end{rmrk}

\begin{lemm}
    \lipsum[][1-4]
\end{lemm}

\begin{proof}
    \lipsum[1]
\end{proof}

\begin{thrm}[\ind{Thrm Ipsum}]
    \lipsum[][1-4]
\end{thrm}

\begin{proof}
    \lipsum[1]
\end{proof}

\begin{coro}
    \lipsum[][1-4]
\end{coro}

\section{Section}

\lipsum[1-2]

%%%%%%%%%%%%%%%%%%%%%%%
\subsection{Subsection}

\lipsum[][1-9]

\begin{defn}[\ind{Defn Ipsum}]
    \lipsum[][1] \(\sum_{i=1}^{n} n = \frac{n(n+1)}{2}\) \lipsum[][2-3]
    \[e^{i\pi}+1 = 0\]
    \lipsum[][4] \(\int_{a}^{b} x^2 \ dx = \frac{x^3}{3} + C\) \lipsum[][5-6]
\end{defn}

\begin{nota}
    \lipsum[][1-3]
    \[\sum_{i=1}^{n} n = \frac{n(n+1)}{2}\]
    \lipsum[][4-5]
    \[\int x \ dx = \frac{x^2}{2} + C\]
    \lipsum[][6-7]
\end{nota}

\begin{defn}
    %Given any natural number \(n\):
    \[\sum_{i=1}^{n} n = \frac{n(n+1)}{2}\]
    \lipsum[][3-5]
\end{defn}

\begin{rmrk}
    No environment should ever start with inline- or (especially not) display-mode math. Not only is that \href{https://kconrad.math.uconn.edu/blurbs/proofs/writingtips.pdf}{bad writing practice} but, in the case of starting with display-mode math such as above, blank vertical space will be left before the display-mode math where text is expected to be.

    If you don't know what to write, just state some context about the equation or expression with which you intended to start, e.g. `Given any natural number \(n\):'.
\end{rmrk}

\begin{exmp}[This is transcribed from Example 1.3 \href{https://kconrad.math.uconn.edu/blurbs/proofs/welldefined.pdf}{here}]
    In calculus, \(\int_{a}^{b} f(x) \ dx\) can be computed as
    \[\int_{a}^{b} f(x) \ dx = F(b) - F(a),\]
    where \(F(x)\) is an arbitrary anti-derivative of \(f(x)\) on \([a,b]\), \textit{i.e.}, \(F'(x) = f(x)\) for all \(x\) in \([a,b]\). This formula for \(\int_{a}^{b} f(x) \ dx\) involves a choice of anti-derivative for \(f(x)\), but the formula does \textit{not} depend on the choice: every anti-derivative \(G(x)\) of \(f(x)\) on \([a,b]\) differs from \(F(x)\) by a constant, say \(G(x) = F(x) + C\) for all \(x\) in \([a,b]\), and changing the anti-derivative \(G(x)\) does not change the difference of its values at the endpoints:
    \[G(b) - G(a) = (F(b) + C) - (F(a) + C) = F(b) - F(a).\]
    So the difference of the values of an anti-derivative of \(f(x)\) at \(x=a\) and \(x=b\) is independent of the choice of anti-derivative of \(f(x)\) on the interval \([a,b]\).\footnote[2]{This is why in physics, potential energy has no intrinsic meaning (the zero level of potential energy can be anywhere), but \textit{differences} in potential energy are physically meaningful.}

    In contrast, the ``rule'' \(F(b)+F(a)\) depends on the choice of anti-derivative of \(f(x)\), since
    \[G(b)+G(a) = (F(b)+ C) + (F(a) + C) = F(b) + F(a) + 2C,\]
    which is a new value if \(C \neq 0\). Taking differences in an anti-derivative cancels the effect of the undetermined additive constant, so the expression \(F(b)-F(a)\) is a well-defined value based on the original input function \(f(x)\) and the interval \([a,b]\).
\end{exmp}

\begin{note}
    \lipsum[][1-9]
\end{note}

\begin{rmrk}
    \lipsum[][1-9]
\end{rmrk}

\begin{lemm}
    \lipsum[][1-4]
\end{lemm}

\begin{proof}
    \lipsum[1]
\end{proof}

\begin{thrm}[\ind{Thrm Ipsum}]
    \lipsum[][1-4]
\end{thrm}

\begin{proof}
    \lipsum[1]
\end{proof}

\begin{coro}
    \lipsum[][1-4]
\end{coro}

\begin{exrc}
    \lipsum[1]
\end{exrc}

\begin{soln}
    \lipsum[1]
\end{soln}

%%%%%%%%%%%%%%%%%%%%%%%%%%%%%
\subsubsection{Subsubsection}

\begin{defn}[\ind{Defn Ipsum}]
    \lipsum[][1] \(\sum_{i=1}^{n} n = \frac{n(n+1)}{2}\) \lipsum[][2-3]
    \[e^{i\pi}+1 = 0\]
    \lipsum[][4] \(\int_{a}^{b} x^2 \ dx = \frac{x^3}{3} + C\) \lipsum[][5-6]
\end{defn}

\begin{nota}
    \lipsum[][1-3]
    \[\sum_{i=1}^{n} n = \frac{n(n+1)}{2}\]
    \lipsum[][4-5]
    \[\int x \ dx = \frac{x^2}{2} + C\]
    \lipsum[][6-7]
\end{nota}

\begin{defn}
    %Given any natural number \(n\):
    \[\sum_{i=1}^{n} n = \frac{n(n+1)}{2}\]
    \lipsum[][3-5]
\end{defn}

\begin{rmrk}
    No environment should ever start with inline- or (especially not) display-mode math. Not only is that \href{https://kconrad.math.uconn.edu/blurbs/proofs/writingtips.pdf}{bad writing practice} but, in the case of starting with display-mode math such as above, blank vertical space will be left before the display-mode math where text is expected to be.

    If you don't know what to write, just state some context about the equation or expression with which you intended to start, e.g. `Given any natural number \(n\):'.
\end{rmrk}

\begin{exmp}[This is transcribed from Example 1.3 \href{https://kconrad.math.uconn.edu/blurbs/proofs/welldefined.pdf}{here}]
    In calculus, \(\int_{a}^{b} f(x) \ dx\) can be computed as
    \[\int_{a}^{b} f(x) \ dx = F(b) - F(a),\]
    where \(F(x)\) is an arbitrary anti-derivative of \(f(x)\) on \([a,b]\), \textit{i.e.}, \(F'(x) = f(x)\) for all \(x\) in \([a,b]\). This formula for \(\int_{a}^{b} f(x) \ dx\) involves a choice of anti-derivative for \(f(x)\), but the formula does \textit{not} depend on the choice: every anti-derivative \(G(x)\) of \(f(x)\) on \([a,b]\) differs from \(F(x)\) by a constant, say \(G(x) = F(x) + C\) for all \(x\) in \([a,b]\), and changing the anti-derivative \(G(x)\) does not change the difference of its values at the endpoints:
    \[G(b) - G(a) = (F(b) + C) - (F(a) + C) = F(b) - F(a).\]
    So the difference of the values of an anti-derivative of \(f(x)\) at \(x=a\) and \(x=b\) is independent of the choice of anti-derivative of \(f(x)\) on the interval \([a,b]\).\footnote[2]{This is why in physics, potential energy has no intrinsic meaning (the zero level of potential energy can be anywhere), but \textit{differences} in potential energy are physically meaningful.}

    In contrast, the ``rule'' \(F(b)+F(a)\) depends on the choice of anti-derivative of \(f(x)\), since
    \[G(b)+G(a) = (F(b)+ C) + (F(a) + C) = F(b) + F(a) + 2C,\]
    which is a new value if \(C \neq 0\). Taking differences in an anti-derivative cancels the effect of the undetermined additive constant, so the expression \(F(b)-F(a)\) is a well-defined value based on the original input function \(f(x)\) and the interval \([a,b]\).
\end{exmp}

\begin{note}
    \lipsum[][1-9]
\end{note}

\begin{rmrk}
    \lipsum[][1-9]
\end{rmrk}

\begin{lemm}
    \lipsum[][1-4]
\end{lemm}

\begin{proof}
    \lipsum[1]
\end{proof}

\begin{thrm}[\ind{Thrm Ipsum}]
    \lipsum[][1-4]
\end{thrm}

\begin{proof}
    \lipsum[1]
\end{proof}

\begin{coro}
    \lipsum[][1-4]
\end{coro}

%%%%%%%%%%%%%%%%%%%%%%%%%%%%%
\subsubsection{Subsubsection}

\begin{defn}[\ind{Defn Ipsum}]
    \lipsum[][1] \(\sum_{i=1}^{n} n = \frac{n(n+1)}{2}\) \lipsum[][2-3]
    \[e^{i\pi}+1 = 0\]
    \lipsum[][4] \(\int_{a}^{b} x^2 \ dx = \frac{x^3}{3} + C\) \lipsum[][5-6]
\end{defn}

\begin{nota}
    \lipsum[][1-3]
    \[\sum_{i=1}^{n} n = \frac{n(n+1)}{2}\]
    \lipsum[][4-5]
    \[\int x \ dx = \frac{x^2}{2} + C\]
    \lipsum[][6-7]
\end{nota}

\begin{defn}
    %Given any natural number \(n\):
    \[\sum_{i=1}^{n} n = \frac{n(n+1)}{2}\]
    \lipsum[][3-5]
\end{defn}

\begin{rmrk}
    No environment should ever start with inline- or (especially not) display-mode math. Not only is that \href{https://kconrad.math.uconn.edu/blurbs/proofs/writingtips.pdf}{bad writing practice} but, in the case of starting with display-mode math such as above, blank vertical space will be left before the display-mode math where text is expected to be.

    If you don't know what to write, just state some context about the equation or expression with which you intended to start, e.g. `Given any natural number \(n\):'.
\end{rmrk}

\begin{exmp}[This is transcribed from Example 1.3 \href{https://kconrad.math.uconn.edu/blurbs/proofs/welldefined.pdf}{here}]
    In calculus, \(\int_{a}^{b} f(x) \ dx\) can be computed as
    \[\int_{a}^{b} f(x) \ dx = F(b) - F(a),\]
    where \(F(x)\) is an arbitrary anti-derivative of \(f(x)\) on \([a,b]\), \textit{i.e.}, \(F'(x) = f(x)\) for all \(x\) in \([a,b]\). This formula for \(\int_{a}^{b} f(x) \ dx\) involves a choice of anti-derivative for \(f(x)\), but the formula does \textit{not} depend on the choice: every anti-derivative \(G(x)\) of \(f(x)\) on \([a,b]\) differs from \(F(x)\) by a constant, say \(G(x) = F(x) + C\) for all \(x\) in \([a,b]\), and changing the anti-derivative \(G(x)\) does not change the difference of its values at the endpoints:
    \[G(b) - G(a) = (F(b) + C) - (F(a) + C) = F(b) - F(a).\]
    So the difference of the values of an anti-derivative of \(f(x)\) at \(x=a\) and \(x=b\) is independent of the choice of anti-derivative of \(f(x)\) on the interval \([a,b]\).\footnote[2]{This is why in physics, potential energy has no intrinsic meaning (the zero level of potential energy can be anywhere), but \textit{differences} in potential energy are physically meaningful.}

    In contrast, the ``rule'' \(F(b)+F(a)\) depends on the choice of anti-derivative of \(f(x)\), since
    \[G(b)+G(a) = (F(b)+ C) + (F(a) + C) = F(b) + F(a) + 2C,\]
    which is a new value if \(C \neq 0\). Taking differences in an anti-derivative cancels the effect of the undetermined additive constant, so the expression \(F(b)-F(a)\) is a well-defined value based on the original input function \(f(x)\) and the interval \([a,b]\).
\end{exmp}

\begin{note}
    \lipsum[][1-9]
\end{note}

\begin{rmrk}
    \lipsum[][1-9]
\end{rmrk}

\begin{lemm}
    \lipsum[][1-4]
\end{lemm}

\begin{proof}
    \lipsum[1]
\end{proof}

\begin{thrm}[\ind{Thrm Ipsum}]
    \lipsum[][1-4]
\end{thrm}

\begin{proof}
    \lipsum[1]
\end{proof}

\begin{coro}
    \lipsum[][1-4]
\end{coro}

%%%%%%%%%%%%%%%%%%%%%%%%%%%%%
\subsubsection{Subsubsection}

\begin{defn}[\ind{Defn Ipsum}]
    \lipsum[][1] \(\sum_{i=1}^{n} n = \frac{n(n+1)}{2}\) \lipsum[][2-3]
    \[e^{i\pi}+1 = 0\]
    \lipsum[][4] \(\int_{a}^{b} x^2 \ dx = \frac{x^3}{3} + C\) \lipsum[][5-6]
\end{defn}

\begin{nota}
    \lipsum[][1-3]
    \[\sum_{i=1}^{n} n = \frac{n(n+1)}{2}\]
    \lipsum[][4-5]
    \[\int x \ dx = \frac{x^2}{2} + C\]
    \lipsum[][6-7]
\end{nota}

\begin{defn}
    %Given any natural number \(n\):
    \[\sum_{i=1}^{n} n = \frac{n(n+1)}{2}\]
    \lipsum[][3-5]
\end{defn}

\begin{rmrk}
    No environment should ever start with inline- or (especially not) display-mode math. Not only is that \href{https://kconrad.math.uconn.edu/blurbs/proofs/writingtips.pdf}{bad writing practice} but, in the case of starting with display-mode math such as above, blank vertical space will be left before the display-mode math where text is expected to be.

    If you don't know what to write, just state some context about the equation or expression with which you intended to start, e.g. `Given any natural number \(n\):'.
\end{rmrk}

\begin{exmp}[This is transcribed from Example 1.3 \href{https://kconrad.math.uconn.edu/blurbs/proofs/welldefined.pdf}{here}]
    In calculus, \(\int_{a}^{b} f(x) \ dx\) can be computed as
    \[\int_{a}^{b} f(x) \ dx = F(b) - F(a),\]
    where \(F(x)\) is an arbitrary anti-derivative of \(f(x)\) on \([a,b]\), \textit{i.e.}, \(F'(x) = f(x)\) for all \(x\) in \([a,b]\). This formula for \(\int_{a}^{b} f(x) \ dx\) involves a choice of anti-derivative for \(f(x)\), but the formula does \textit{not} depend on the choice: every anti-derivative \(G(x)\) of \(f(x)\) on \([a,b]\) differs from \(F(x)\) by a constant, say \(G(x) = F(x) + C\) for all \(x\) in \([a,b]\), and changing the anti-derivative \(G(x)\) does not change the difference of its values at the endpoints:
    \[G(b) - G(a) = (F(b) + C) - (F(a) + C) = F(b) - F(a).\]
    So the difference of the values of an anti-derivative of \(f(x)\) at \(x=a\) and \(x=b\) is independent of the choice of anti-derivative of \(f(x)\) on the interval \([a,b]\).\footnote[2]{This is why in physics, potential energy has no intrinsic meaning (the zero level of potential energy can be anywhere), but \textit{differences} in potential energy are physically meaningful.}

    In contrast, the ``rule'' \(F(b)+F(a)\) depends on the choice of anti-derivative of \(f(x)\), since
    \[G(b)+G(a) = (F(b)+ C) + (F(a) + C) = F(b) + F(a) + 2C,\]
    which is a new value if \(C \neq 0\). Taking differences in an anti-derivative cancels the effect of the undetermined additive constant, so the expression \(F(b)-F(a)\) is a well-defined value based on the original input function \(f(x)\) and the interval \([a,b]\).
\end{exmp}

\begin{note}
    \lipsum[][1-9]
\end{note}

\begin{rmrk}
    \lipsum[][1-9]
\end{rmrk}

\begin{lemm}
    \lipsum[][1-4]
\end{lemm}

\begin{proof}
    \lipsum[1]
\end{proof}

\begin{thrm}[\ind{Thrm Ipsum}]
    \lipsum[][1-4]
\end{thrm}

\begin{proof}
    \lipsum[1]
\end{proof}

\begin{coro}
    \lipsum[][1-4]
\end{coro}

%%%%%%%%%%%%%%%%%%%%%%%
\subsection{Subsection}

\lipsum[][1-9]

\begin{defn}[\ind{Defn Ipsum}]
    \lipsum[][1] \(\sum_{i=1}^{n} n = \frac{n(n+1)}{2}\) \lipsum[][2-3]
    \[e^{i\pi}+1 = 0\]
    \lipsum[][4] \(\int_{a}^{b} x^2 \ dx = \frac{x^3}{3} + C\) \lipsum[][5-6]
\end{defn}

\begin{nota}
    \lipsum[][1-3]
    \[\sum_{i=1}^{n} n = \frac{n(n+1)}{2}\]
    \lipsum[][4-5]
    \[\int x \ dx = \frac{x^2}{2} + C\]
    \lipsum[][6-7]
\end{nota}

\begin{defn}
    %Given any natural number \(n\):
    \[\sum_{i=1}^{n} n = \frac{n(n+1)}{2}\]
    \lipsum[][3-5]
\end{defn}

\begin{rmrk}
    No environment should ever start with inline- or (especially not) display-mode math. Not only is that \href{https://kconrad.math.uconn.edu/blurbs/proofs/writingtips.pdf}{bad writing practice} but, in the case of starting with display-mode math such as above, blank vertical space will be left before the display-mode math where text is expected to be.

    If you don't know what to write, just state some context about the equation or expression with which you intended to start, e.g. `Given any natural number \(n\):'.
\end{rmrk}

\begin{exmp}[This is transcribed from Example 1.3 \href{https://kconrad.math.uconn.edu/blurbs/proofs/welldefined.pdf}{here}]
    In calculus, \(\int_{a}^{b} f(x) \ dx\) can be computed as
    \[\int_{a}^{b} f(x) \ dx = F(b) - F(a),\]
    where \(F(x)\) is an arbitrary anti-derivative of \(f(x)\) on \([a,b]\), \textit{i.e.}, \(F'(x) = f(x)\) for all \(x\) in \([a,b]\). This formula for \(\int_{a}^{b} f(x) \ dx\) involves a choice of anti-derivative for \(f(x)\), but the formula does \textit{not} depend on the choice: every anti-derivative \(G(x)\) of \(f(x)\) on \([a,b]\) differs from \(F(x)\) by a constant, say \(G(x) = F(x) + C\) for all \(x\) in \([a,b]\), and changing the anti-derivative \(G(x)\) does not change the difference of its values at the endpoints:
    \[G(b) - G(a) = (F(b) + C) - (F(a) + C) = F(b) - F(a).\]
    So the difference of the values of an anti-derivative of \(f(x)\) at \(x=a\) and \(x=b\) is independent of the choice of anti-derivative of \(f(x)\) on the interval \([a,b]\).\footnote[2]{This is why in physics, potential energy has no intrinsic meaning (the zero level of potential energy can be anywhere), but \textit{differences} in potential energy are physically meaningful.}

    In contrast, the ``rule'' \(F(b)+F(a)\) depends on the choice of anti-derivative of \(f(x)\), since
    \[G(b)+G(a) = (F(b)+ C) + (F(a) + C) = F(b) + F(a) + 2C,\]
    which is a new value if \(C \neq 0\). Taking differences in an anti-derivative cancels the effect of the undetermined additive constant, so the expression \(F(b)-F(a)\) is a well-defined value based on the original input function \(f(x)\) and the interval \([a,b]\).
\end{exmp}

\begin{note}
    \lipsum[][1-9]
\end{note}

\begin{rmrk}
    \lipsum[][1-9]
\end{rmrk}

\begin{lemm}
    \lipsum[][1-4]
\end{lemm}

\begin{proof}
    \lipsum[1]
\end{proof}

\begin{thrm}[\ind{Thrm Ipsum}]
    \lipsum[][1-4]
\end{thrm}

\begin{proof}
    \lipsum[1]
\end{proof}

\begin{coro}
    \lipsum[][1-4]
\end{coro}

\begin{exrc}
    \lipsum[1]
\end{exrc}

\begin{soln}
    \lipsum[1]
\end{soln}

%%%%%%%%%%%%%%%%%%%%%%%%%%%%%
\subsubsection{Subsubsection}

\begin{defn}[\ind{Defn Ipsum}]
    \lipsum[][1] \(\sum_{i=1}^{n} n = \frac{n(n+1)}{2}\) \lipsum[][2-3]
    \[e^{i\pi}+1 = 0\]
    \lipsum[][4] \(\int_{a}^{b} x^2 \ dx = \frac{x^3}{3} + C\) \lipsum[][5-6]
\end{defn}

\begin{nota}
    \lipsum[][1-3]
    \[\sum_{i=1}^{n} n = \frac{n(n+1)}{2}\]
    \lipsum[][4-5]
    \[\int x \ dx = \frac{x^2}{2} + C\]
    \lipsum[][6-7]
\end{nota}

\begin{defn}
    %Given any natural number \(n\):
    \[\sum_{i=1}^{n} n = \frac{n(n+1)}{2}\]
    \lipsum[][3-5]
\end{defn}

\begin{rmrk}
    No environment should ever start with inline- or (especially not) display-mode math. Not only is that \href{https://kconrad.math.uconn.edu/blurbs/proofs/writingtips.pdf}{bad writing practice} but, in the case of starting with display-mode math such as above, blank vertical space will be left before the display-mode math where text is expected to be.

    If you don't know what to write, just state some context about the equation or expression with which you intended to start, e.g. `Given any natural number \(n\):'.
\end{rmrk}

\begin{exmp}[This is transcribed from Example 1.3 \href{https://kconrad.math.uconn.edu/blurbs/proofs/welldefined.pdf}{here}]
    In calculus, \(\int_{a}^{b} f(x) \ dx\) can be computed as
    \[\int_{a}^{b} f(x) \ dx = F(b) - F(a),\]
    where \(F(x)\) is an arbitrary anti-derivative of \(f(x)\) on \([a,b]\), \textit{i.e.}, \(F'(x) = f(x)\) for all \(x\) in \([a,b]\). This formula for \(\int_{a}^{b} f(x) \ dx\) involves a choice of anti-derivative for \(f(x)\), but the formula does \textit{not} depend on the choice: every anti-derivative \(G(x)\) of \(f(x)\) on \([a,b]\) differs from \(F(x)\) by a constant, say \(G(x) = F(x) + C\) for all \(x\) in \([a,b]\), and changing the anti-derivative \(G(x)\) does not change the difference of its values at the endpoints:
    \[G(b) - G(a) = (F(b) + C) - (F(a) + C) = F(b) - F(a).\]
    So the difference of the values of an anti-derivative of \(f(x)\) at \(x=a\) and \(x=b\) is independent of the choice of anti-derivative of \(f(x)\) on the interval \([a,b]\).\footnote[2]{This is why in physics, potential energy has no intrinsic meaning (the zero level of potential energy can be anywhere), but \textit{differences} in potential energy are physically meaningful.}

    In contrast, the ``rule'' \(F(b)+F(a)\) depends on the choice of anti-derivative of \(f(x)\), since
    \[G(b)+G(a) = (F(b)+ C) + (F(a) + C) = F(b) + F(a) + 2C,\]
    which is a new value if \(C \neq 0\). Taking differences in an anti-derivative cancels the effect of the undetermined additive constant, so the expression \(F(b)-F(a)\) is a well-defined value based on the original input function \(f(x)\) and the interval \([a,b]\).
\end{exmp}

\begin{note}
    \lipsum[][1-9]
\end{note}

\begin{rmrk}
    \lipsum[][1-9]
\end{rmrk}

\begin{lemm}
    \lipsum[][1-4]
\end{lemm}

\begin{proof}
    \lipsum[1]
\end{proof}

\begin{thrm}[\ind{Thrm Ipsum}]
    \lipsum[][1-4]
\end{thrm}

\begin{proof}
    \lipsum[1]
\end{proof}

\begin{coro}
    \lipsum[][1-4]
\end{coro}

%%%%%%%%%%%%%%%%%%%%%%%%%%%%%
\subsubsection{Subsubsection}

\begin{defn}[\ind{Defn Ipsum}]
    \lipsum[][1] \(\sum_{i=1}^{n} n = \frac{n(n+1)}{2}\) \lipsum[][2-3]
    \[e^{i\pi}+1 = 0\]
    \lipsum[][4] \(\int_{a}^{b} x^2 \ dx = \frac{x^3}{3} + C\) \lipsum[][5-6]
\end{defn}

\begin{nota}
    \lipsum[][1-3]
    \[\sum_{i=1}^{n} n = \frac{n(n+1)}{2}\]
    \lipsum[][4-5]
    \[\int x \ dx = \frac{x^2}{2} + C\]
    \lipsum[][6-7]
\end{nota}

\begin{defn}
    %Given any natural number \(n\):
    \[\sum_{i=1}^{n} n = \frac{n(n+1)}{2}\]
    \lipsum[][3-5]
\end{defn}

\begin{rmrk}
    No environment should ever start with inline- or (especially not) display-mode math. Not only is that \href{https://kconrad.math.uconn.edu/blurbs/proofs/writingtips.pdf}{bad writing practice} but, in the case of starting with display-mode math such as above, blank vertical space will be left before the display-mode math where text is expected to be.

    If you don't know what to write, just state some context about the equation or expression with which you intended to start, e.g. `Given any natural number \(n\):'.
\end{rmrk}

\begin{exmp}[This is transcribed from Example 1.3 \href{https://kconrad.math.uconn.edu/blurbs/proofs/welldefined.pdf}{here}]
    In calculus, \(\int_{a}^{b} f(x) \ dx\) can be computed as
    \[\int_{a}^{b} f(x) \ dx = F(b) - F(a),\]
    where \(F(x)\) is an arbitrary anti-derivative of \(f(x)\) on \([a,b]\), \textit{i.e.}, \(F'(x) = f(x)\) for all \(x\) in \([a,b]\). This formula for \(\int_{a}^{b} f(x) \ dx\) involves a choice of anti-derivative for \(f(x)\), but the formula does \textit{not} depend on the choice: every anti-derivative \(G(x)\) of \(f(x)\) on \([a,b]\) differs from \(F(x)\) by a constant, say \(G(x) = F(x) + C\) for all \(x\) in \([a,b]\), and changing the anti-derivative \(G(x)\) does not change the difference of its values at the endpoints:
    \[G(b) - G(a) = (F(b) + C) - (F(a) + C) = F(b) - F(a).\]
    So the difference of the values of an anti-derivative of \(f(x)\) at \(x=a\) and \(x=b\) is independent of the choice of anti-derivative of \(f(x)\) on the interval \([a,b]\).\footnote[2]{This is why in physics, potential energy has no intrinsic meaning (the zero level of potential energy can be anywhere), but \textit{differences} in potential energy are physically meaningful.}

    In contrast, the ``rule'' \(F(b)+F(a)\) depends on the choice of anti-derivative of \(f(x)\), since
    \[G(b)+G(a) = (F(b)+ C) + (F(a) + C) = F(b) + F(a) + 2C,\]
    which is a new value if \(C \neq 0\). Taking differences in an anti-derivative cancels the effect of the undetermined additive constant, so the expression \(F(b)-F(a)\) is a well-defined value based on the original input function \(f(x)\) and the interval \([a,b]\).
\end{exmp}

\begin{note}
    \lipsum[][1-9]
\end{note}

\begin{rmrk}
    \lipsum[][1-9]
\end{rmrk}

\begin{lemm}
    \lipsum[][1-4]
\end{lemm}

\begin{proof}
    \lipsum[1]
\end{proof}

\begin{thrm}[\ind{Thrm Ipsum}]
    \lipsum[][1-4]
\end{thrm}

\begin{proof}
    \lipsum[1]
\end{proof}

\begin{coro}
    \lipsum[][1-4]
\end{coro}

%%%%%%%%%%%%%%%%%%%%%%%%%%%%%
\subsubsection{Subsubsection}

\begin{defn}[\ind{Defn Ipsum}]
    \lipsum[][1] \(\sum_{i=1}^{n} n = \frac{n(n+1)}{2}\) \lipsum[][2-3]
    \[e^{i\pi}+1 = 0\]
    \lipsum[][4] \(\int_{a}^{b} x^2 \ dx = \frac{x^3}{3} + C\) \lipsum[][5-6]
\end{defn}

\begin{nota}
    \lipsum[][1-3]
    \[\sum_{i=1}^{n} n = \frac{n(n+1)}{2}\]
    \lipsum[][4-5]
    \[\int x \ dx = \frac{x^2}{2} + C\]
    \lipsum[][6-7]
\end{nota}

\begin{defn}
    %Given any natural number \(n\):
    \[\sum_{i=1}^{n} n = \frac{n(n+1)}{2}\]
    \lipsum[][3-5]
\end{defn}

\begin{rmrk}
    No environment should ever start with inline- or (especially not) display-mode math. Not only is that \href{https://kconrad.math.uconn.edu/blurbs/proofs/writingtips.pdf}{bad writing practice} but, in the case of starting with display-mode math such as above, blank vertical space will be left before the display-mode math where text is expected to be.

    If you don't know what to write, just state some context about the equation or expression with which you intended to start, e.g. `Given any natural number \(n\):'.
\end{rmrk}

\begin{exmp}[This is transcribed from Example 1.3 \href{https://kconrad.math.uconn.edu/blurbs/proofs/welldefined.pdf}{here}]
    In calculus, \(\int_{a}^{b} f(x) \ dx\) can be computed as
    \[\int_{a}^{b} f(x) \ dx = F(b) - F(a),\]
    where \(F(x)\) is an arbitrary anti-derivative of \(f(x)\) on \([a,b]\), \textit{i.e.}, \(F'(x) = f(x)\) for all \(x\) in \([a,b]\). This formula for \(\int_{a}^{b} f(x) \ dx\) involves a choice of anti-derivative for \(f(x)\), but the formula does \textit{not} depend on the choice: every anti-derivative \(G(x)\) of \(f(x)\) on \([a,b]\) differs from \(F(x)\) by a constant, say \(G(x) = F(x) + C\) for all \(x\) in \([a,b]\), and changing the anti-derivative \(G(x)\) does not change the difference of its values at the endpoints:
    \[G(b) - G(a) = (F(b) + C) - (F(a) + C) = F(b) - F(a).\]
    So the difference of the values of an anti-derivative of \(f(x)\) at \(x=a\) and \(x=b\) is independent of the choice of anti-derivative of \(f(x)\) on the interval \([a,b]\).\footnote[2]{This is why in physics, potential energy has no intrinsic meaning (the zero level of potential energy can be anywhere), but \textit{differences} in potential energy are physically meaningful.}

    In contrast, the ``rule'' \(F(b)+F(a)\) depends on the choice of anti-derivative of \(f(x)\), since
    \[G(b)+G(a) = (F(b)+ C) + (F(a) + C) = F(b) + F(a) + 2C,\]
    which is a new value if \(C \neq 0\). Taking differences in an anti-derivative cancels the effect of the undetermined additive constant, so the expression \(F(b)-F(a)\) is a well-defined value based on the original input function \(f(x)\) and the interval \([a,b]\).
\end{exmp}

\begin{note}
    \lipsum[][1-9]
\end{note}

\begin{rmrk}
    \lipsum[][1-9]
\end{rmrk}

\begin{lemm}
    \lipsum[][1-4]
\end{lemm}

\begin{proof}
    \lipsum[1]
\end{proof}

\begin{thrm}[\ind{Thrm Ipsum}]
    \lipsum[][1-4]
\end{thrm}

\begin{proof}
    \lipsum[1]
\end{proof}

\begin{coro}
    \lipsum[][1-4]
\end{coro}

%%%%%%%%%%%%%%%%%%%%%%%
\subsection{Subsection}

\lipsum[][1-9]

\begin{defn}[\ind{Defn Ipsum}]
    \lipsum[][1] \(\sum_{i=1}^{n} n = \frac{n(n+1)}{2}\) \lipsum[][2-3]
    \[e^{i\pi}+1 = 0\]
    \lipsum[][4] \(\int_{a}^{b} x^2 \ dx = \frac{x^3}{3} + C\) \lipsum[][5-6]
\end{defn}

\begin{nota}
    \lipsum[][1-3]
    \[\sum_{i=1}^{n} n = \frac{n(n+1)}{2}\]
    \lipsum[][4-5]
    \[\int x \ dx = \frac{x^2}{2} + C\]
    \lipsum[][6-7]
\end{nota}

\begin{defn}
    %Given any natural number \(n\):
    \[\sum_{i=1}^{n} n = \frac{n(n+1)}{2}\]
    \lipsum[][3-5]
\end{defn}

\begin{rmrk}
    No environment should ever start with inline- or (especially not) display-mode math. Not only is that \href{https://kconrad.math.uconn.edu/blurbs/proofs/writingtips.pdf}{bad writing practice} but, in the case of starting with display-mode math such as above, blank vertical space will be left before the display-mode math where text is expected to be.

    If you don't know what to write, just state some context about the equation or expression with which you intended to start, e.g. `Given any natural number \(n\):'.
\end{rmrk}

\begin{exmp}[This is transcribed from Example 1.3 \href{https://kconrad.math.uconn.edu/blurbs/proofs/welldefined.pdf}{here}]
    In calculus, \(\int_{a}^{b} f(x) \ dx\) can be computed as
    \[\int_{a}^{b} f(x) \ dx = F(b) - F(a),\]
    where \(F(x)\) is an arbitrary anti-derivative of \(f(x)\) on \([a,b]\), \textit{i.e.}, \(F'(x) = f(x)\) for all \(x\) in \([a,b]\). This formula for \(\int_{a}^{b} f(x) \ dx\) involves a choice of anti-derivative for \(f(x)\), but the formula does \textit{not} depend on the choice: every anti-derivative \(G(x)\) of \(f(x)\) on \([a,b]\) differs from \(F(x)\) by a constant, say \(G(x) = F(x) + C\) for all \(x\) in \([a,b]\), and changing the anti-derivative \(G(x)\) does not change the difference of its values at the endpoints:
    \[G(b) - G(a) = (F(b) + C) - (F(a) + C) = F(b) - F(a).\]
    So the difference of the values of an anti-derivative of \(f(x)\) at \(x=a\) and \(x=b\) is independent of the choice of anti-derivative of \(f(x)\) on the interval \([a,b]\).\footnote[2]{This is why in physics, potential energy has no intrinsic meaning (the zero level of potential energy can be anywhere), but \textit{differences} in potential energy are physically meaningful.}

    In contrast, the ``rule'' \(F(b)+F(a)\) depends on the choice of anti-derivative of \(f(x)\), since
    \[G(b)+G(a) = (F(b)+ C) + (F(a) + C) = F(b) + F(a) + 2C,\]
    which is a new value if \(C \neq 0\). Taking differences in an anti-derivative cancels the effect of the undetermined additive constant, so the expression \(F(b)-F(a)\) is a well-defined value based on the original input function \(f(x)\) and the interval \([a,b]\).
\end{exmp}

\begin{note}
    \lipsum[][1-9]
\end{note}

\begin{rmrk}
    \lipsum[][1-9]
\end{rmrk}

\begin{lemm}
    \lipsum[][1-4]
\end{lemm}

\begin{proof}
    \lipsum[1]
\end{proof}

\begin{thrm}[\ind{Thrm Ipsum}]
    \lipsum[][1-4]
\end{thrm}

\begin{proof}
    \lipsum[1]
\end{proof}

\begin{coro}
    \lipsum[][1-4]
\end{coro}

\begin{exrc}
    \lipsum[1]
\end{exrc}

\begin{soln}
    \lipsum[1]
\end{soln}

%%%%%%%%%%%%%%%%%%%%%%%%%%%%%
\subsubsection{Subsubsection}

\begin{defn}[\ind{Defn Ipsum}]
    \lipsum[][1] \(\sum_{i=1}^{n} n = \frac{n(n+1)}{2}\) \lipsum[][2-3]
    \[e^{i\pi}+1 = 0\]
    \lipsum[][4] \(\int_{a}^{b} x^2 \ dx = \frac{x^3}{3} + C\) \lipsum[][5-6]
\end{defn}

\begin{nota}
    \lipsum[][1-3]
    \[\sum_{i=1}^{n} n = \frac{n(n+1)}{2}\]
    \lipsum[][4-5]
    \[\int x \ dx = \frac{x^2}{2} + C\]
    \lipsum[][6-7]
\end{nota}

\begin{defn}
    %Given any natural number \(n\):
    \[\sum_{i=1}^{n} n = \frac{n(n+1)}{2}\]
    \lipsum[][3-5]
\end{defn}

\begin{rmrk}
    No environment should ever start with inline- or (especially not) display-mode math. Not only is that \href{https://kconrad.math.uconn.edu/blurbs/proofs/writingtips.pdf}{bad writing practice} but, in the case of starting with display-mode math such as above, blank vertical space will be left before the display-mode math where text is expected to be.

    If you don't know what to write, just state some context about the equation or expression with which you intended to start, e.g. `Given any natural number \(n\):'.
\end{rmrk}

\begin{exmp}[This is transcribed from Example 1.3 \href{https://kconrad.math.uconn.edu/blurbs/proofs/welldefined.pdf}{here}]
    In calculus, \(\int_{a}^{b} f(x) \ dx\) can be computed as
    \[\int_{a}^{b} f(x) \ dx = F(b) - F(a),\]
    where \(F(x)\) is an arbitrary anti-derivative of \(f(x)\) on \([a,b]\), \textit{i.e.}, \(F'(x) = f(x)\) for all \(x\) in \([a,b]\). This formula for \(\int_{a}^{b} f(x) \ dx\) involves a choice of anti-derivative for \(f(x)\), but the formula does \textit{not} depend on the choice: every anti-derivative \(G(x)\) of \(f(x)\) on \([a,b]\) differs from \(F(x)\) by a constant, say \(G(x) = F(x) + C\) for all \(x\) in \([a,b]\), and changing the anti-derivative \(G(x)\) does not change the difference of its values at the endpoints:
    \[G(b) - G(a) = (F(b) + C) - (F(a) + C) = F(b) - F(a).\]
    So the difference of the values of an anti-derivative of \(f(x)\) at \(x=a\) and \(x=b\) is independent of the choice of anti-derivative of \(f(x)\) on the interval \([a,b]\).\footnote[2]{This is why in physics, potential energy has no intrinsic meaning (the zero level of potential energy can be anywhere), but \textit{differences} in potential energy are physically meaningful.}

    In contrast, the ``rule'' \(F(b)+F(a)\) depends on the choice of anti-derivative of \(f(x)\), since
    \[G(b)+G(a) = (F(b)+ C) + (F(a) + C) = F(b) + F(a) + 2C,\]
    which is a new value if \(C \neq 0\). Taking differences in an anti-derivative cancels the effect of the undetermined additive constant, so the expression \(F(b)-F(a)\) is a well-defined value based on the original input function \(f(x)\) and the interval \([a,b]\).
\end{exmp}

\begin{note}
    \lipsum[][1-9]
\end{note}

\begin{rmrk}
    \lipsum[][1-9]
\end{rmrk}

\begin{lemm}
    \lipsum[][1-4]
\end{lemm}

\begin{proof}
    \lipsum[1]
\end{proof}

\begin{thrm}[\ind{Thrm Ipsum}]
    \lipsum[][1-4]
\end{thrm}

\begin{proof}
    \lipsum[1]
\end{proof}

\begin{coro}
    \lipsum[][1-4]
\end{coro}

%%%%%%%%%%%%%%%%%%%%%%%%%%%%%
\subsubsection{Subsubsection}

\begin{defn}[\ind{Defn Ipsum}]
    \lipsum[][1] \(\sum_{i=1}^{n} n = \frac{n(n+1)}{2}\) \lipsum[][2-3]
    \[e^{i\pi}+1 = 0\]
    \lipsum[][4] \(\int_{a}^{b} x^2 \ dx = \frac{x^3}{3} + C\) \lipsum[][5-6]
\end{defn}

\begin{nota}
    \lipsum[][1-3]
    \[\sum_{i=1}^{n} n = \frac{n(n+1)}{2}\]
    \lipsum[][4-5]
    \[\int x \ dx = \frac{x^2}{2} + C\]
    \lipsum[][6-7]
\end{nota}

\begin{defn}
    %Given any natural number \(n\):
    \[\sum_{i=1}^{n} n = \frac{n(n+1)}{2}\]
    \lipsum[][3-5]
\end{defn}

\begin{rmrk}
    No environment should ever start with inline- or (especially not) display-mode math. Not only is that \href{https://kconrad.math.uconn.edu/blurbs/proofs/writingtips.pdf}{bad writing practice} but, in the case of starting with display-mode math such as above, blank vertical space will be left before the display-mode math where text is expected to be.

    If you don't know what to write, just state some context about the equation or expression with which you intended to start, e.g. `Given any natural number \(n\):'.
\end{rmrk}

\begin{exmp}[This is transcribed from Example 1.3 \href{https://kconrad.math.uconn.edu/blurbs/proofs/welldefined.pdf}{here}]
    In calculus, \(\int_{a}^{b} f(x) \ dx\) can be computed as
    \[\int_{a}^{b} f(x) \ dx = F(b) - F(a),\]
    where \(F(x)\) is an arbitrary anti-derivative of \(f(x)\) on \([a,b]\), \textit{i.e.}, \(F'(x) = f(x)\) for all \(x\) in \([a,b]\). This formula for \(\int_{a}^{b} f(x) \ dx\) involves a choice of anti-derivative for \(f(x)\), but the formula does \textit{not} depend on the choice: every anti-derivative \(G(x)\) of \(f(x)\) on \([a,b]\) differs from \(F(x)\) by a constant, say \(G(x) = F(x) + C\) for all \(x\) in \([a,b]\), and changing the anti-derivative \(G(x)\) does not change the difference of its values at the endpoints:
    \[G(b) - G(a) = (F(b) + C) - (F(a) + C) = F(b) - F(a).\]
    So the difference of the values of an anti-derivative of \(f(x)\) at \(x=a\) and \(x=b\) is independent of the choice of anti-derivative of \(f(x)\) on the interval \([a,b]\).\footnote[2]{This is why in physics, potential energy has no intrinsic meaning (the zero level of potential energy can be anywhere), but \textit{differences} in potential energy are physically meaningful.}

    In contrast, the ``rule'' \(F(b)+F(a)\) depends on the choice of anti-derivative of \(f(x)\), since
    \[G(b)+G(a) = (F(b)+ C) + (F(a) + C) = F(b) + F(a) + 2C,\]
    which is a new value if \(C \neq 0\). Taking differences in an anti-derivative cancels the effect of the undetermined additive constant, so the expression \(F(b)-F(a)\) is a well-defined value based on the original input function \(f(x)\) and the interval \([a,b]\).
\end{exmp}

\begin{note}
    \lipsum[][1-9]
\end{note}

\begin{rmrk}
    \lipsum[][1-9]
\end{rmrk}

\begin{lemm}
    \lipsum[][1-4]
\end{lemm}

\begin{proof}
    \lipsum[1]
\end{proof}

\begin{thrm}[\ind{Thrm Ipsum}]
    \lipsum[][1-4]
\end{thrm}

\begin{proof}
    \lipsum[1]
\end{proof}

\begin{coro}
    \lipsum[][1-4]
\end{coro}

%%%%%%%%%%%%%%%%%%%%%%%%%%%%%
\subsubsection{Subsubsection}

\begin{defn}[\ind{Defn Ipsum}]
    \lipsum[][1] \(\sum_{i=1}^{n} n = \frac{n(n+1)}{2}\) \lipsum[][2-3]
    \[e^{i\pi}+1 = 0\]
    \lipsum[][4] \(\int_{a}^{b} x^2 \ dx = \frac{x^3}{3} + C\) \lipsum[][5-6]
\end{defn}

\begin{nota}
    \lipsum[][1-3]
    \[\sum_{i=1}^{n} n = \frac{n(n+1)}{2}\]
    \lipsum[][4-5]
    \[\int x \ dx = \frac{x^2}{2} + C\]
    \lipsum[][6-7]
\end{nota}

\begin{defn}
    %Given any natural number \(n\):
    \[\sum_{i=1}^{n} n = \frac{n(n+1)}{2}\]
    \lipsum[][3-5]
\end{defn}

\begin{rmrk}
    No environment should ever start with inline- or (especially not) display-mode math. Not only is that \href{https://kconrad.math.uconn.edu/blurbs/proofs/writingtips.pdf}{bad writing practice} but, in the case of starting with display-mode math such as above, blank vertical space will be left before the display-mode math where text is expected to be.

    If you don't know what to write, just state some context about the equation or expression with which you intended to start, e.g. `Given any natural number \(n\):'.
\end{rmrk}

\begin{exmp}[This is transcribed from Example 1.3 \href{https://kconrad.math.uconn.edu/blurbs/proofs/welldefined.pdf}{here}]
    In calculus, \(\int_{a}^{b} f(x) \ dx\) can be computed as
    \[\int_{a}^{b} f(x) \ dx = F(b) - F(a),\]
    where \(F(x)\) is an arbitrary anti-derivative of \(f(x)\) on \([a,b]\), \textit{i.e.}, \(F'(x) = f(x)\) for all \(x\) in \([a,b]\). This formula for \(\int_{a}^{b} f(x) \ dx\) involves a choice of anti-derivative for \(f(x)\), but the formula does \textit{not} depend on the choice: every anti-derivative \(G(x)\) of \(f(x)\) on \([a,b]\) differs from \(F(x)\) by a constant, say \(G(x) = F(x) + C\) for all \(x\) in \([a,b]\), and changing the anti-derivative \(G(x)\) does not change the difference of its values at the endpoints:
    \[G(b) - G(a) = (F(b) + C) - (F(a) + C) = F(b) - F(a).\]
    So the difference of the values of an anti-derivative of \(f(x)\) at \(x=a\) and \(x=b\) is independent of the choice of anti-derivative of \(f(x)\) on the interval \([a,b]\).\footnote[2]{This is why in physics, potential energy has no intrinsic meaning (the zero level of potential energy can be anywhere), but \textit{differences} in potential energy are physically meaningful.}

    In contrast, the ``rule'' \(F(b)+F(a)\) depends on the choice of anti-derivative of \(f(x)\), since
    \[G(b)+G(a) = (F(b)+ C) + (F(a) + C) = F(b) + F(a) + 2C,\]
    which is a new value if \(C \neq 0\). Taking differences in an anti-derivative cancels the effect of the undetermined additive constant, so the expression \(F(b)-F(a)\) is a well-defined value based on the original input function \(f(x)\) and the interval \([a,b]\).
\end{exmp}

\begin{note}
    \lipsum[][1-9]
\end{note}

\begin{rmrk}
    \lipsum[][1-9]
\end{rmrk}

\begin{lemm}
    \lipsum[][1-4]
\end{lemm}

\begin{proof}
    \lipsum[1]
\end{proof}

\begin{thrm}[\ind{Thrm Ipsum}]
    \lipsum[][1-4]
\end{thrm}

\begin{proof}
    \lipsum[1]
\end{proof}

\begin{coro}
    \lipsum[][1-4]
\end{coro}

